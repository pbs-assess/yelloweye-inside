% Documents setup
\documentclass[french,11pt]{book}

% fix for pandoc 1.14
\providecommand{\tightlist}{%
  \setlength{\itemsep}{0pt}\setlength{\parskip}{0pt}}

\usepackage{tabu} % https://tex.stackexchange.com/questions/50332/vertical-spacing-of-a-table-cell

% Location of the csas-style repository: adjust path as needed
\newcommand{\locRepo}{csas-style}

% Use the style file in the csas-style repository (res-doc.sty)
\usepackage{\locRepo/res-doc-french}

% header-includes from R markdown entry


% Headers and footers
\lhead{}
% \lhead{}
\rhead{}
% \rfoot{DRAFT - DO NOT CITE}

%%%% Commands for title page etc %%%%%

% Publication year
\newcommand{\rdYear}{2021}

% Publication month
\newcommand{\rdMonth}{}

% Report number
\newcommand{\rdNumber}{8}

% Region
\newcommand{\rdRegion}{Pacific Region}

% Title
\newcommand{\rdTitle}{Évaluation des stratégies de rétablissement possibles pour le sébaste aux yeux jaunes (\emph{Sebastes ruberrimus}) des eaux intérieures de la Colombie-Britannique}

\newcommand{\rdISBN}{Fs70-5/2021-008F-PDF}
\newcommand{\rdCatNo}{978-0-660-38699-7}

% Author names separated by commas and ', and' for the last author in the format 'M.H. Grinnell' (use \textsuperscript{n} for addresses)
\newcommand{\rdAuth}{Dana R. Haggarty\textsuperscript{1}, Quang C. Huynh\textsuperscript{2}, Robyn E. Forrest\textsuperscript{1}, Sean C. Anderson\textsuperscript{1}, Midoli J. Bresch\textsuperscript{1}, Elise A. Keppel\textsuperscript{1}}

% Author names reversed separated by commas in the format 'Grinnell, M.H.'
\newcommand{\rdAuthRev}{Haggarty, D.R., C.R. Huynh, R.E. Forrest, S.C. Anderson, M.J. Bresch, et E.A. Keppel}

% Author addresses (use \textsuperscript{n})
\newcommand{\rdAuthAddy}{\textsuperscript{1}Station biologique du Pacifique\\
Pêches et Océans Canada, 3190, chemin Hammond Bay\\
Nanaimo (Colombie-Britannique) V9T 6N7, Canada\\
\textsuperscript{2}Institut pour les océans et la pêche\\
LRAE de l'Université de la Colombie- Britannique, 2202, Main Mall\\
Vancouver (Colombie-Britannique) V6T 1Z4, Canada\\}

\newcommand{\citationOtherLanguage}{Haggarty, D.R., Huynh, Q.C., Forrest, R.E., Anderson, S.C., Bresch, M.J., Keppel, E.A. 2021. Evaluation of potential rebuilding strategies for Inside Yelloweye Rockfish (\emph{Sebastes ruberrimus}) in British Columbia. DFO Can. Sci. Advis. Sec. Res. Doc. 2021/008. vi + 141 p.}

% Name of file with abstract and resume (see \abstract and \frenchabstract for requirements)
\newcommand{\rdAbstract}{\abstract{En vertu des politiques et de la législation canadiennes, il faut Eétablir un plan de rétablissement pour les stocks de poissons qui ont été évalués comme étant inférieurs au point de référence limite (PRL) afin de les ramener au-delà du PRL. Les plans de rétablissement doivent être fondés sur des objectifs caractérisés par 1) une cible, 2) un délai souhaité pour atteindre la cible et 3) une probabilité acceptable d'atteindre la cible. Les plans de rétablissement doivent également comprendre des mesures de gestion ou des procédures de gestion, des jalons cibles et être évalués régulièrement. \vspace{1.5mm} \break Le stock de sébaste aux yeux jaunes (\emph{Sebastes ruberrimus}) des eaux intérieures est un stock sur lequel on dispose de données limitées, présent dans la zone de gestion du poisson de fond 4B (détroit de la Reine-Charlotte, détroit de Georgie et détroit de Juan de Fuca) en Colombie-Britannique. Il a été évalué comme étant inférieur au PLR en 2010, ce qui a donné lieu à la publication d'un plan de rétablissement. Il est également inscrit en vertu de la \emph{Loi sur les espèces en péril} comme espèce préoccupante. L'actuelle procédure de gestion pour assurer le rétablissement est un total autorisé des captures (TAC) annuel fixe de 15 tonnes métriques, qui n'a pas été réévalué depuis la dernière évaluation. \vspace{1.5mm} \break Ce projet vise à fournir un avis scientifique à l'appui de la réévaluation du plan de rétablissement du sébaste aux yeux jaunes des eaux intérieures. Nous appliquons un nouveau cadre d'évaluation de la stratégie de gestion (le Cadre des procédures de gestion), récemment élaboré pour le poisson de fond de la Colombie-Britannique, afin d'évaluer le rendement des autres procédures de gestion à données limitées pour ce qui est de l'atteinte des objectifs de rétablissement. Le Cadre des procédures de gestion suit six étapes de pratiques exemplaires pour évaluer la stratégie de gestion~: 1) la définition du contexte décisionnel; 2) l'établissement des objectifs et des paramètres de rendement; 3) la précision des modèles opérationnels pour représenter le système sous-jacent et calculer les paramètres de rendement; 4) la sélection des procédures de gestion possibles; 5) la réalisation de simulations en boucle fermée afin d'évaluer le rendement des procédures de gestion; 6) la présentation des résultats pour faciliter l'évaluation des compromis. \vspace{1.5mm} \break Nous avons appliqué ce cadre pour évaluer le rendement de 34 procédures de gestion à données limitées pour ce qui est de l'atteinte de l'objectif principal, qui est de ramener le stock au-dessus du PRL sur 1,5 génération avec au moins une probabilité de réussite de 95 \% {[}19 fois sur 20{]}. Nous avons également évalué le rendement des procédures de gestion en ce qui concerne deux autres paramètres de conservation, quatre objectifs de prises moyennes et un objectif de variabilité des prises. Pour tenir compte de l'incertitude liée à la dynamique de la population sous-jacente et aux sources de données, nous avons élaboré six scénarios de modèles opérationnels de rechange, qui différaient de par les hypothèses précises du modèle et des données. Ces scénarios de modèles opérationnels ont été divisés en un « ensemble de référence » (quatre modèles opérationnels) et un « ensemble de robustesse » (deux modèles opérationnels). Nous avons conditionné tous les modèles opérationnels aux données sur les prises observées, aux indices de l'abondance et aux données accessibles sur la composition selon l'âge. Nous avons utilisé la simulation en boucle fermée pour évaluer le rendement des procédures de gestion et nous avons éliminé celles qui ne satisfaisaient pas à un ensemble de critères de base, ce qui a laissé cinq procédures de gestion possibles~: des procédures de gestion à prises constantes annuelles de 10 ou 15 tonnes et trois procédures de gestion qui ajustent le TAC en fonction de la pente relative de l'indice de l'abondance dans le relevé à la palangre sur fond dur dans les eaux intérieures. \vspace{1.5mm} \break Les cinq procédures de gestion finales atteignaient l'objectif principal avec une probabilité supérieure à 0,98 (49 fois sur 50), dans les scénarios des quatre modèles opérationnels de l'ensemble de référence, surtout qu'aucun des modèles opérationnels de l'ensemble de référence n'a estimé que le stock serait inférieur au PRL en 2020. Dans les scénarios des deux modèles opérationnels de l'ensemble de robustesse, le scénario qui simulait une plus grande variabilité dans le futur relevé à la palangre sur fond dur a donné des résultats semblables à ceux des scénarios de l'ensemble de référence. Cependant, dans le scénario qui supposait un taux de mortalité naturelle plus faible pour le stock (« M faible »), toutes les procédures de gestion avaient des probabilités plus basses d'atteindre l'objectif principal, la probabilité la plus faible étant atteinte par la procédure de gestion actuelle (prises constantes de 15 tonnes). \vspace{1.5mm} \break Nous présentons un certain nombre de visualisations pour illustrer les compromis entre les objectifs de conservation et de prises pour les différentes procédures de gestion dans d'autres scénarios de modèles opérationnels. Ces visualisations présentent les compromis sous forme de tableaux et de graphiques, destinés à faciliter le processus de sélection de la procédure de gestion finale. Étant donné que toutes les procédures de gestion ont atteint l'objectif principal dans les scénarios de l'ensemble de référence, il n'y avait pas de compromis important entre les objectifs de conservation et les objectifs de prises. Parmi les deux scénarios de l'ensemble de robustesse, les compromis étaient les plus évidents dans le scénario de M faible, où la probabilité d'atteindre l'objectif principal diminuait à mesure que la probabilité de prises moyennes à court terme de 10 tonnes augmentait. \vspace{1.5mm} \break Nous discutons des incertitudes majeures, y compris l'incertitude entourant la mortalité naturelle, la sélectivité et les prises historiques, en notant que nous avons tenté d'en tenir compte en évaluant le rendement des procédures de gestion dans plusieurs modèles opérationnels. Nous soulignons les problèmes concernant les estimations de l'état actuel du stock de sébaste aux yeux jaunes des eaux intérieures et le rôle des points de référence dans le Cadre des procédures de gestion. Nous formulons des recommandations sur la fréquence des évaluations et suggérons des déclencheurs pour la réévaluation. Nous évaluons également le rendement des procédures de gestion en ce qui concerne le respect de deux autres critères d'évaluation pour le Comité sur la situation des espèces en péril au Canada.}}

%%%% End of title page commands %%%%%

% \pdfcompresslevel=5 % faster PNGs

\setcounter{section}{0}

\bibliographystyle{csas-style/res-doc}

\usepackage{amsmath}
\usepackage{bm}

% commands and environments needed by pandoc snippets
% extracted from the output of `pandoc -s`
%% Make R markdown code chunks work
\usepackage{array}
\usepackage{amssymb,amsmath}
\usepackage{color}
\usepackage{fancyvrb}
% From default template:
\newcommand{\VerbBar}{|}
\newcommand{\VERB}{\Verb[commandchars=\\\{\}]}
\DefineVerbatimEnvironment{Highlighting}{Verbatim}{commandchars=\\\{\}}
% Add ',fontsize=\small' for more characters per line
\usepackage{framed}
\definecolor{shadecolor}{RGB}{248,248,248}
\newenvironment{Shaded}{\begin{snugshade}}{\end{snugshade}}
\newcommand{\AlertTok}[1]{\textcolor[rgb]{0.94,0.16,0.16}{#1}}
\newcommand{\AnnotationTok}[1]{\textcolor[rgb]{0.56,0.35,0.01}{\textbf{\textit{#1}}}}
\newcommand{\AttributeTok}[1]{\textcolor[rgb]{0.77,0.63,0.00}{#1}}
\newcommand{\BaseNTok}[1]{\textcolor[rgb]{0.00,0.00,0.81}{#1}}
\newcommand{\BuiltInTok}[1]{#1}
\newcommand{\CharTok}[1]{\textcolor[rgb]{0.31,0.60,0.02}{#1}}
\newcommand{\CommentTok}[1]{\textcolor[rgb]{0.56,0.35,0.01}{\textit{#1}}}
\newcommand{\CommentVarTok}[1]{\textcolor[rgb]{0.56,0.35,0.01}{\textbf{\textit{#1}}}}
\newcommand{\ConstantTok}[1]{\textcolor[rgb]{0.00,0.00,0.00}{#1}}
\newcommand{\ControlFlowTok}[1]{\textcolor[rgb]{0.13,0.29,0.53}{\textbf{#1}}}
\newcommand{\DataTypeTok}[1]{\textcolor[rgb]{0.13,0.29,0.53}{#1}}
\newcommand{\DecValTok}[1]{\textcolor[rgb]{0.00,0.00,0.81}{#1}}
\newcommand{\DocumentationTok}[1]{\textcolor[rgb]{0.56,0.35,0.01}{\textbf{\textit{#1}}}}
\newcommand{\ErrorTok}[1]{\textcolor[rgb]{0.64,0.00,0.00}{\textbf{#1}}}
\newcommand{\ExtensionTok}[1]{#1}
\newcommand{\FloatTok}[1]{\textcolor[rgb]{0.00,0.00,0.81}{#1}}
\newcommand{\FunctionTok}[1]{\textcolor[rgb]{0.00,0.00,0.00}{#1}}
\newcommand{\ImportTok}[1]{#1}
\newcommand{\InformationTok}[1]{\textcolor[rgb]{0.56,0.35,0.01}{\textbf{\textit{#1}}}}
\newcommand{\KeywordTok}[1]{\textcolor[rgb]{0.13,0.29,0.53}{\textbf{#1}}}
\newcommand{\NormalTok}[1]{#1}
\newcommand{\OperatorTok}[1]{\textcolor[rgb]{0.81,0.36,0.00}{\textbf{#1}}}
\newcommand{\OtherTok}[1]{\textcolor[rgb]{0.56,0.35,0.01}{#1}}
\newcommand{\PreprocessorTok}[1]{\textcolor[rgb]{0.56,0.35,0.01}{\textit{#1}}}
\newcommand{\RegionMarkerTok}[1]{#1}
\newcommand{\SpecialCharTok}[1]{\textcolor[rgb]{0.00,0.00,0.00}{#1}}
\newcommand{\SpecialStringTok}[1]{\textcolor[rgb]{0.31,0.60,0.02}{#1}}
\newcommand{\StringTok}[1]{\textcolor[rgb]{0.31,0.60,0.02}{#1}}
\newcommand{\VariableTok}[1]{\textcolor[rgb]{0.00,0.00,0.00}{#1}}
\newcommand{\VerbatimStringTok}[1]{\textcolor[rgb]{0.31,0.60,0.02}{#1}}
\newcommand{\WarningTok}[1]{\textcolor[rgb]{0.56,0.35,0.01}{\textbf{\textit{#1}}}}

\newcommand{\lt}{\ensuremath <}
\newcommand{\gt}{\ensuremath >}

%Defines cslreferences environment
%Required by pandoc 2.8
%Copied from https://github.com/rstudio/rmarkdown/issues/1649

\DeclareGraphicsExtensions{.png,.pdf}
\begin{document}
\renewcommand{\tablename}{Tableau}
\frontmatter

\section{FACTEURS CONSIDÉRÉS PAR LE COSEPAC}
\label{app:cosewic}

Le stock de sébaste aux yeux jaunes des eaux intérieures est également inscrit en vertu de la \emph{Loi sur les espèces en péril} (LEP) comme espèce préoccupante (COSEWIC \protect\hyperlink{ref-cosewic2008}{2008}) et il est prévu que le COSEPAC le réévaluera en 2020. Le COSEPAC et le MPO ont des critères différents pour évaluer la situation des stocks de poissons marins. Le MPO se concentre sur l'état actuel comparativement à un état ou un seuil de référence, tandis que les critères du COSEPAC (fondés sur les catégories de la Liste rouge de l'UICN) mettent l'accent sur le déclin observé au cours des générations passées et sur la probabilité de déclins continus dans le futur (COSEWIC \protect\hyperlink{ref-cosewic2015}{2015}). Le COSEPAC applique une série de critères d'évaluation quantitatifs et de lignes directrices pour élaborer et attribuer un état au stock en question. Afin de guider la réévaluation du sébaste aux yeux jaunes des eaux intérieures, nous présentons des résultats pour deux critères d'évaluation quantitatifs du COSEPAC qui pourraient s'appliquer à ce stock, soit la mesure A et la mesure E.

\hypertarget{mesure-a-du-cosepac}{%
\subsection{MESURE A DU COSEPAC}\label{mesure-a-du-cosepac}}

La mesure A du COSEPAC permet de calculer la probabilité que le stock ait diminué de 70 \%, 50 \% ou 30 \% après trois générations, où la valeur d'une génération de sébaste aux yeux jaunes des eaux intérieures est établie à 38 ans. Il convient de noter qu'en vertu de cette définition, trois générations équivalent à 114 ans, mais puisque notre période historique (de 1918 à 2019) est de 102 ans, nous utilisons cette valeur comme approximation pour trois générations. Ces seuils de probabilité servent à attribuer des désignations d'état aux espèces en voie de disparition, menacées ou préoccupantes, respectivement, même si d'autres facteurs, comme la cause du déclin, sont aussi considérés (COSEWIC \protect\hyperlink{ref-cosewic2015}{2015}). Pour guider la réévaluation du sébaste aux yeux jaunes des eaux intérieures par le COSEPAC, nous présentons l'information suivante pour chaque modèle opérationnel (figure~\ref{fig:cosewic-metrics})~:
\begin{enumerate}
\def\labelenumi{\arabic{enumi}.}

\item
  P70 -- Probabilité que le stock ait décliné, en moyenne, de plus de 70 \% par rapport à \emph{B}\textsubscript{1918} sur trois générations, où la valeur d'une durée de génération est fixée à 38 ans et la probabilité est calculée selon \(P[1 - B_{2019}/B_{1918} > 0,7]\).
\item
  P50 -- Probabilité que le stock ait décliné, en moyenne, de plus de 50 \% par rapport à \emph{B}\textsubscript{1918} sur trois générations.
\item
  P30 -- Probabilité que le stock ait décliné, en moyenne, de plus de 30 \% par rapport à \emph{B}\textsubscript{1918} sur trois générations.
\end{enumerate}

\begin{figure}[htb]

{\centering \pdftooltip{\includegraphics[width=3in]{C:/GitHub/yelloweye-inside/mse/figures-french/historical_indicators2}}{Figure \ref{fig:cosewic-metrics}} 

}

\caption{Résultats pour la mesure A du COSEPAC, soit la probabilité que le stock ait décliné en moyenne de plus de 70 \%, 50 \% ou 30 \% par rapport à \emph{B}\textsubscript{1918} au cours des trois dernières générations, pour chaque scénario de modèle opérationnel, où la valeur d'une durée de génération est établie à 38 ans.}\label{fig:cosewic-metrics}
\end{figure}
\hypertarget{risque-dextinction-mesure-e-du-cosepac}{%
\subsection{RISQUE D'EXTINCTION -- MESURE E DU COSEPAC}\label{risque-dextinction-mesure-e-du-cosepac}}

La mesure E du COSEPAC permet de calculer la probabilité d'une extinction future du stock. Un stock est désigné comme en voie de disparition si la probabilité d'extinction est de 20 \% dans un délai de 20 ans (ou cinq générations, selon lequel est le plus long) et comme menacé si la probabilité d'extinction est de 10 \% dans un délai de 10 ans. Le critère E est rarement appliqué aux poissons marins, car il dépend fortement des données et des hypothèses concernant les paramètres requis comme données d'entrée pour les analyses de viabilité des populations (Ross Claytor, COSEPAC, comm. pers., 29 janvier 2020). Une autre exigence pour estimer le risque d'extinction consiste à établir le seuil d'extinction, qui n'est pas explicitement défini dans les critères du COSEPAC (COSEWIC \protect\hyperlink{ref-cosewic2015}{2015}).

Pour évaluer la probabilité d'une extinction future en fonction des procédures de gestion proposées, des seuils d'extinction spécifiques au stock doivent être attribués. En deçà de ces seuils, le stock serait considéré comme effectivement disparu du pays ou disparu. Nous proposons deux seuils d'extinction possibles de 2 \%\emph{B}\textsubscript{0} et 5 \%\emph{B}\textsubscript{0}. Ces seuils arbitraires ont été établis en fonction de précédents dans la littérature (p.~ex., Forrest et al. \protect\hyperlink{ref-forrest2015}{2015} a utilisé 5 \%\emph{B}\textsubscript{0}) et d'estimations historiques du déclin pour d'autres espèces dans le détroit de Georgie; par exemple, on estime que le stock de morue-lingue dans le détroit de Georgie a décliné à un seuil aussi bas que 2 \%\emph{B}\textsubscript{0}, mais qu'il est en voie de rétablissement (Logan et al. \protect\hyperlink{ref-logan2005}{2005}). Dans le futur, des essais par simulation de seuils proposés pourraient servir à déterminer les seuils d'extinction spécifiques à un stock.

En utilisant une période de projection de 100 ans, nous avons calculé la probabilité qu'en moyenne, au cours de cette période de projection de 100 ans, le stock demeure au-dessus de 2 \% et de 5 \% de \emph{B}\textsubscript{0} avec les cinq procédures de gestion satisfaisantes et la procédure de gestion de référence où la population n'est pas exploitée, pour chaque scénario individuel de modèle opérationnel (figure~\ref{fig:cosewic-all}). Nous avons également calculé la probabilité moyenne que le stock demeure au-dessus de chaque seuil avec ces procédures de gestion pour tous les scénarios de l'ensemble de référence de modèles opérationnels (figure~\ref{fig:cosewic-avg})~:
\begin{enumerate}
\def\labelenumi{\arabic{enumi}.}

\item
  2 \%B\textsubscript{0} = \(P[B_y > 0,02 B_0]\) en moyenne durant la totalité de la période de projection.
\item
  5 \%B\textsubscript{0} = \([B_y > 0,05 B_0]\) en moyenne durant la totalité de la période de projection.
\end{enumerate}
Les résultats de cette analyse montrent qu'en utilisant ces seuils, le risque d'extinction est très faible avec les procédures de gestion satisfaisantes.


\begin{figure}[htb]

{\centering \pdftooltip{\includegraphics[width=\textwidth]{C:/GitHub/yelloweye-inside/mse/figures-french/ye-tigure-cosewic-all}}{Figure \ref{fig:cosewic-all}} 

}

\caption{Probabilité que le stock de sébaste aux yeux jaunes des eaux intérieures demeure supérieur à 2 \% et à 5 \% de \emph{B}\textsubscript{0}, par scénario de modèle opérationnel et procédure de gestion.}\label{fig:cosewic-all}
\end{figure}
\clearpage


\begin{figure}[htb]

{\centering \pdftooltip{\includegraphics[width=3in]{C:/GitHub/yelloweye-inside/mse/figures-french/ye-tigure-cosewic-avg}}{Figure \ref{fig:cosewic-avg}} 

}

\caption{Probabilité que le stock de sébaste aux yeux jaunes des eaux intérieures demeure supérieur à 2 \% et à 5 \% de \emph{B}\textsubscript{0}, en moyenne pour les modèles opérationnels de l'ensemble de référence.}\label{fig:cosewic-avg}
\end{figure}
\clearpage

\hypertarget{environnement-informatique}{%
\section{ENVIRONNEMENT INFORMATIQUE}\label{environnement-informatique}}

Cette version du document a été produite sur 2021-10-29 12:20:04 avec R version 3.6.3 (2020-02-29) (R Core Team \protect\hyperlink{ref-r2019}{2019}) et des versions du progiciel R:
\begin{longtable}[]{@{}llll@{}}
\toprule
& Package & Version & Date \\
\midrule
\endhead
bookdown & bookdown & 0.24 & 2021-09-02 \\
cowplot & cowplot & 1.1.1 & 2020-12-30 \\
csasdown & csasdown & 0.0.10.9000 & 2021-08-13 \\
DLMtool & DLMtool & 5.4.3 & 2021-10-26 \\
dplyr & dplyr & 1.0.7 & 2021-06-18 \\
gfdata & gfdata & 0.0.0.9000 & 2021-07-05 \\
gfdlm & gfdlm & 0.0.1.9001 & 2021-10-26 \\
gfplot & gfplot & 0.1.4 & 2021-08-13 \\
ggplot2 & ggplot2 & 3.3.5 & 2021-06-25 \\
knitr & knitr & 1.36 & 2021-09-29 \\
MSEtool & MSEtool & 1.6.0 & 2020-05-05 \\
purrr & purrr & 0.3.4 & 2020-04-17 \\
rmarkdown & rmarkdown & 2.11 & 2021-09-14 \\
tidyr & tidyr & 1.1.4 & 2021-09-27 \\
TMB & TMB & 1.7.22 & 2021-09-28 \\
\bottomrule
\end{longtable}
\vspace{4mm}

\vspace{4mm}

Le code source de ce document est accessible à l'adresse suivante:\\
\url{https://github.com/pbs-assess/yelloweye-inside/tree/2f9a8a4}.

Ce document a été compilé avec le progiciel csasdown en R (Anderson et al. \protect\hyperlink{ref-csasdown}{2020}).

Les versions particulières des logiciels de base utilisés pour générer ce rapport peuvent être consultées aux adresses:

\url{https://github.com/DLMtool/DLMtool/tree/fa971cf}\\
\url{https://github.com/tcarruth/MSEtool/tree/fa1498c}~\\
\url{https://github.com/pbs-assess/gfdata/tree/7292039}~\\
\url{https://github.com/pbs-assess/gfplot/tree/e0b36c0}~\\
\url{https://github.com/pbs-assess/gfdlm/tree/b895686}~\\
\url{https://github.com/pbs-assess/csasdown/tree/f9d5081}~\\

\clearpage

\hypertarget{refs}{}
\leavevmode\hypertarget{ref-csasdown}{}%
Anderson, S.C., Grandin, C., Edwards, A.M., Grinnell, M.H., Ricard, D., et Haigh, R. 2020. csasdown: Reproducible CSAS Reports with bookdown. R package version 0.0.8. \url{https://github.com/pbs-assess/csasdown}.

\leavevmode\hypertarget{ref-cosewic2008}{}%
COSEWIC. 2008. COSEWIC assessment and status report on the Yelloweye Rockfish (\emph{Sebastes ruberrimus}), Pacific Ocean inside waters population and Pacific Ocean outside waters population, in Canada. Committee on the Status of Endangered Wildlife in Canada \url{https://www.sararegistry.gc.ca/virtual_sara/files/cosewic/sr_yelloweye_rockfish_0809_e.pdf}.

\leavevmode\hypertarget{ref-cosewic2015}{}%
COSEWIC. 2015. COSEWIC Assessment Process, Categories and Guidelines. Committee on the Status of Endangered Wildlife in Canada \url{http://cosewic.ca/index.php/en-ca/assessment-process/wildlife-species-assessment-process-categories-guidelines/quantitative-criteria}.

\leavevmode\hypertarget{ref-forrest2015}{}%
Forrest, R.E., Savina, M., Fulton, E.A., et Pitcher, T.J. 2015. Do marine ecosystem models give consistent policy evaluations? A comparison of Atlantis and Ecosim. Fish. Res. 167: 293‑312.

\leavevmode\hypertarget{ref-logan2005}{}%
Logan, G., Mare, W. de la, King, J., et Haggarty, D. 2005. Management framework for Strait of Georgia Lingcod. DFO Can. Sci. Advis. Sec. Res. Doc. (2005/048).

\leavevmode\hypertarget{ref-r2019}{}%
R Core Team. 2019. R: A Language and Environment for Statistical Computing. R Foundation for Statistical Computing, Vienna, Austria.

\end{document}
