% Documents setup
\documentclass[11pt]{book}

% fix for pandoc 1.14
\providecommand{\tightlist}{%
  \setlength{\itemsep}{0pt}\setlength{\parskip}{0pt}}

\usepackage{tabu} % https://tex.stackexchange.com/questions/50332/vertical-spacing-of-a-table-cell

% Location of the csas-style repository: adjust path as needed
\newcommand{\locRepo}{csas-style}

% Use the style file in the csas-style repository (res-doc.sty)
\usepackage{\locRepo/res-doc}

% header-includes from R markdown entry


% Headers and footers
\lhead{}
% \lhead{}
\rhead{}
% \rfoot{DRAFT - DO NOT CITE}

%%%% Commands for title page etc %%%%%

% Publication year
\newcommand{\rdYear}{2021}

% Publication month
\newcommand{\rdMonth}{}

% Report number
\newcommand{\rdNumber}{8}

% Region
\newcommand{\rdRegion}{Pacific Region}

% Title
\newcommand{\rdTitle}{Évaluation des stratégies de rétablissement possibles pour le sébaste aux yeux jaunes (\emph{Sebastes ruberrimus}) des eaux intérieures de la Colombie-Britannique}

\newcommand{\rdISBN}{Fs70-5/2021-008F-PDF}
\newcommand{\rdCatNo}{978-0-660-38699-7}

% Author names separated by commas and ', and' for the last author in the format 'M.H. Grinnell' (use \textsuperscript{n} for addresses)
\newcommand{\rdAuth}{Dana R. Haggarty\textsuperscript{1}, Quang C. Huynh\textsuperscript{2}, Robyn E. Forrest\textsuperscript{1}, Sean C. Anderson\textsuperscript{1}, Midoli J. Bresch\textsuperscript{1}, Elise A. Keppel\textsuperscript{1}}

% Author names reversed separated by commas in the format 'Grinnell, M.H.'
\newcommand{\rdAuthRev}{Haggarty, D.R., C.R. Huynh, R.E. Forrest, S.C. Anderson, M.J. Bresch, et E.A. Keppel}

% Author addresses (use \textsuperscript{n})
\newcommand{\rdAuthAddy}{\textsuperscript{1}Station biologique du Pacifique\\
Pêches et Océans Canada, 3190, chemin Hammond Bay\\
Nanaimo (Colombie-Britannique) V9T 6N7, Canada\\
\textsuperscript{2}Institut pour les océans et la pêche\\
LRAE de l'Université de la Colombie- Britannique, 2202, Main Mall\\
Vancouver (Colombie-Britannique) V6T 1Z4, Canada\\}

\newcommand{\citationOtherLanguage}{Haggarty, D.R., Huynh, Q.C., Forrest, R.E., Anderson, S.C., Bresch, M.J., Keppel, E.A. 2021. Evaluation of potential rebuilding strategies for Inside Yelloweye Rockfish (\emph{Sebastes ruberrimus}) in British Columbia. DFO Can. Sci. Advis. Sec. Res. Doc. 2021/008. vi + 141 p.}

% Name of file with abstract and resume (see \abstract and \frenchabstract for requirements)
\newcommand{\rdAbstract}{\abstract{En vertu des politiques et de la législation canadiennes, il faut Eétablir un plan de rétablissement pour les stocks de poissons qui ont été évalués comme étant inférieurs au point de référence limite (PRL) afin de les ramener au-delà du PRL. Les plans de rétablissement doivent être fondés sur des objectifs caractérisés par 1) une cible, 2) un délai souhaité pour atteindre la cible et 3) une probabilité acceptable d'atteindre la cible. Les plans de rétablissement doivent également comprendre des mesures de gestion ou des procédures de gestion, des jalons cibles et être évalués régulièrement. \vspace{1.5mm} \break Le stock de sébaste aux yeux jaunes (\emph{Sebastes ruberrimus}) des eaux intérieures est un stock sur lequel on dispose de données limitées, présent dans la zone de gestion du poisson de fond 4B (détroit de la Reine-Charlotte, détroit de Georgie et détroit de Juan de Fuca) en Colombie-Britannique. Il a été évalué comme étant inférieur au PLR en 2010, ce qui a donné lieu à la publication d'un plan de rétablissement. Il est également inscrit en vertu de la \emph{Loi sur les espèces en péril} comme espèce préoccupante. L'actuelle procédure de gestion pour assurer le rétablissement est un total autorisé des captures (TAC) annuel fixe de 15 tonnes métriques, qui n'a pas été réévalué depuis la dernière évaluation. \vspace{1.5mm} \break Ce projet vise à fournir un avis scientifique à l'appui de la réévaluation du plan de rétablissement du sébaste aux yeux jaunes des eaux intérieures. Nous appliquons un nouveau cadre d'évaluation de la stratégie de gestion (le Cadre des procédures de gestion), récemment élaboré pour le poisson de fond de la Colombie-Britannique, afin d'évaluer le rendement des autres procédures de gestion à données limitées pour ce qui est de l'atteinte des objectifs de rétablissement. Le Cadre des procédures de gestion suit six étapes de pratiques exemplaires pour évaluer la stratégie de gestion~: 1) la définition du contexte décisionnel; 2) l'établissement des objectifs et des paramètres de rendement; 3) la précision des modèles opérationnels pour représenter le système sous-jacent et calculer les paramètres de rendement; 4) la sélection des procédures de gestion possibles; 5) la réalisation de simulations en boucle fermée afin d'évaluer le rendement des procédures de gestion; 6) la présentation des résultats pour faciliter l'évaluation des compromis. \vspace{1.5mm} \break Nous avons appliqué ce cadre pour évaluer le rendement de 34 procédures de gestion à données limitées pour ce qui est de l'atteinte de l'objectif principal, qui est de ramener le stock au-dessus du PRL sur 1,5 génération avec au moins une probabilité de réussite de 95 \% {[}19 fois sur 20{]}. Nous avons également évalué le rendement des procédures de gestion en ce qui concerne deux autres paramètres de conservation, quatre objectifs de prises moyennes et un objectif de variabilité des prises. Pour tenir compte de l'incertitude liée à la dynamique de la population sous-jacente et aux sources de données, nous avons élaboré six scénarios de modèles opérationnels de rechange, qui différaient de par les hypothèses précises du modèle et des données. Ces scénarios de modèles opérationnels ont été divisés en un « ensemble de référence » (quatre modèles opérationnels) et un « ensemble de robustesse » (deux modèles opérationnels). Nous avons conditionné tous les modèles opérationnels aux données sur les prises observées, aux indices de l'abondance et aux données accessibles sur la composition selon l'âge. Nous avons utilisé la simulation en boucle fermée pour évaluer le rendement des procédures de gestion et nous avons éliminé celles qui ne satisfaisaient pas à un ensemble de critères de base, ce qui a laissé cinq procédures de gestion possibles~: des procédures de gestion à prises constantes annuelles de 10 ou 15 tonnes et trois procédures de gestion qui ajustent le TAC en fonction de la pente relative de l'indice de l'abondance dans le relevé à la palangre sur fond dur dans les eaux intérieures. \vspace{1.5mm} \break Les cinq procédures de gestion finales atteignaient l'objectif principal avec une probabilité supérieure à 0,98 (49 fois sur 50), dans les scénarios des quatre modèles opérationnels de l'ensemble de référence, surtout qu'aucun des modèles opérationnels de l'ensemble de référence n'a estimé que le stock serait inférieur au PRL en 2020. Dans les scénarios des deux modèles opérationnels de l'ensemble de robustesse, le scénario qui simulait une plus grande variabilité dans le futur relevé à la palangre sur fond dur a donné des résultats semblables à ceux des scénarios de l'ensemble de référence. Cependant, dans le scénario qui supposait un taux de mortalité naturelle plus faible pour le stock (« M faible »), toutes les procédures de gestion avaient des probabilités plus basses d'atteindre l'objectif principal, la probabilité la plus faible étant atteinte par la procédure de gestion actuelle (prises constantes de 15 tonnes). \vspace{1.5mm} \break Nous présentons un certain nombre de visualisations pour illustrer les compromis entre les objectifs de conservation et de prises pour les différentes procédures de gestion dans d'autres scénarios de modèles opérationnels. Ces visualisations présentent les compromis sous forme de tableaux et de graphiques, destinés à faciliter le processus de sélection de la procédure de gestion finale. Étant donné que toutes les procédures de gestion ont atteint l'objectif principal dans les scénarios de l'ensemble de référence, il n'y avait pas de compromis important entre les objectifs de conservation et les objectifs de prises. Parmi les deux scénarios de l'ensemble de robustesse, les compromis étaient les plus évidents dans le scénario de M faible, où la probabilité d'atteindre l'objectif principal diminuait à mesure que la probabilité de prises moyennes à court terme de 10 tonnes augmentait. \vspace{1.5mm} \break Nous discutons des incertitudes majeures, y compris l'incertitude entourant la mortalité naturelle, la sélectivité et les prises historiques, en notant que nous avons tenté d'en tenir compte en évaluant le rendement des procédures de gestion dans plusieurs modèles opérationnels. Nous soulignons les problèmes concernant les estimations de l'état actuel du stock de sébaste aux yeux jaunes des eaux intérieures et le rôle des points de référence dans le Cadre des procédures de gestion. Nous formulons des recommandations sur la fréquence des évaluations et suggérons des déclencheurs pour la réévaluation. Nous évaluons également le rendement des procédures de gestion en ce qui concerne le respect de deux autres critères d'évaluation pour le Comité sur la situation des espèces en péril au Canada.}}

%%%% End of title page commands %%%%%

% \pdfcompresslevel=5 % faster PNGs

\setcounter{section}{0}

\bibliographystyle{csas-style/res-doc}

\usepackage{amsmath}
\usepackage{bm}

% commands and environments needed by pandoc snippets
% extracted from the output of `pandoc -s`
%% Make R markdown code chunks work
\usepackage{array}
\usepackage{amssymb,amsmath}
\usepackage{color}
\usepackage{fancyvrb}

% From default template:
\newcommand{\VerbBar}{|}
\newcommand{\VERB}{\Verb[commandchars=\\\{\}]}
\DefineVerbatimEnvironment{Highlighting}{Verbatim}{commandchars=\\\{\}}
% Add ',fontsize=\small' for more characters per line
\usepackage{framed}
\definecolor{shadecolor}{RGB}{248,248,248}
\newenvironment{Shaded}{\begin{snugshade}}{\end{snugshade}}
\newcommand{\AlertTok}[1]{\textcolor[rgb]{0.94,0.16,0.16}{#1}}
\newcommand{\AnnotationTok}[1]{\textcolor[rgb]{0.56,0.35,0.01}{\textbf{\textit{#1}}}}
\newcommand{\AttributeTok}[1]{\textcolor[rgb]{0.77,0.63,0.00}{#1}}
\newcommand{\BaseNTok}[1]{\textcolor[rgb]{0.00,0.00,0.81}{#1}}
\newcommand{\BuiltInTok}[1]{#1}
\newcommand{\CharTok}[1]{\textcolor[rgb]{0.31,0.60,0.02}{#1}}
\newcommand{\CommentTok}[1]{\textcolor[rgb]{0.56,0.35,0.01}{\textit{#1}}}
\newcommand{\CommentVarTok}[1]{\textcolor[rgb]{0.56,0.35,0.01}{\textbf{\textit{#1}}}}
\newcommand{\ConstantTok}[1]{\textcolor[rgb]{0.00,0.00,0.00}{#1}}
\newcommand{\ControlFlowTok}[1]{\textcolor[rgb]{0.13,0.29,0.53}{\textbf{#1}}}
\newcommand{\DataTypeTok}[1]{\textcolor[rgb]{0.13,0.29,0.53}{#1}}
\newcommand{\DecValTok}[1]{\textcolor[rgb]{0.00,0.00,0.81}{#1}}
\newcommand{\DocumentationTok}[1]{\textcolor[rgb]{0.56,0.35,0.01}{\textbf{\textit{#1}}}}
\newcommand{\ErrorTok}[1]{\textcolor[rgb]{0.64,0.00,0.00}{\textbf{#1}}}
\newcommand{\ExtensionTok}[1]{#1}
\newcommand{\FloatTok}[1]{\textcolor[rgb]{0.00,0.00,0.81}{#1}}
\newcommand{\FunctionTok}[1]{\textcolor[rgb]{0.00,0.00,0.00}{#1}}
\newcommand{\ImportTok}[1]{#1}
\newcommand{\InformationTok}[1]{\textcolor[rgb]{0.56,0.35,0.01}{\textbf{\textit{#1}}}}
\newcommand{\KeywordTok}[1]{\textcolor[rgb]{0.13,0.29,0.53}{\textbf{#1}}}
\newcommand{\NormalTok}[1]{#1}
\newcommand{\OperatorTok}[1]{\textcolor[rgb]{0.81,0.36,0.00}{\textbf{#1}}}
\newcommand{\OtherTok}[1]{\textcolor[rgb]{0.56,0.35,0.01}{#1}}
\newcommand{\PreprocessorTok}[1]{\textcolor[rgb]{0.56,0.35,0.01}{\textit{#1}}}
\newcommand{\RegionMarkerTok}[1]{#1}
\newcommand{\SpecialCharTok}[1]{\textcolor[rgb]{0.00,0.00,0.00}{#1}}
\newcommand{\SpecialStringTok}[1]{\textcolor[rgb]{0.31,0.60,0.02}{#1}}
\newcommand{\StringTok}[1]{\textcolor[rgb]{0.31,0.60,0.02}{#1}}
\newcommand{\VariableTok}[1]{\textcolor[rgb]{0.00,0.00,0.00}{#1}}
\newcommand{\VerbatimStringTok}[1]{\textcolor[rgb]{0.31,0.60,0.02}{#1}}
\newcommand{\WarningTok}[1]{\textcolor[rgb]{0.56,0.35,0.01}{\textbf{\textit{#1}}}}

\newcommand{\lt}{\ensuremath <}
\newcommand{\gt}{\ensuremath >}

%Defines cslreferences environment
%Required by pandoc 2.8
%Copied from https://github.com/rstudio/rmarkdown/issues/1649

\DeclareGraphicsExtensions{.png,.pdf}
\begin{document}

\frontmatter

\hypertarget{sec:om}{%
\section{CHOIX DES INCERTITUDES/SPÉCIFICATION DES MODÈLES OPÉRATIONNELS}\label{sec:om}}

Les modèles opérationnels de l'outil DLMtool sont organisés en quatre composantes principales représentant un réseau hydrographique exploité réel~:
\begin{enumerate}
\def\labelenumi{\arabic{enumi}.}

\item
  la dynamique des populations du stock de poissons (p.~ex., croissance, recrutement, mortalité);
\item
  la dynamique de la pêche (p.~ex., sélectivité, ciblage spatial);
\item
  les processus d'observation (p.~ex., biais et précision des indices des relevés);
\item
  la mise en œuvre de la gestion (p.~ex., dépassement des limites de prises).
\end{enumerate}
Les équations et les paramètres décrivant les quatre composantes des modèles opérationnels sont fournis en détail à l'annexe B de Carruthers and Hordyk (\protect\hyperlink{ref-carruthers2018}{2018}) et à l'annexe A de Anderson et al. (\protect\hyperlink{ref-anderson2020gfmp}{2020}\protect\hyperlink{ref-anderson2020gfmp}{a}). L'outil DLMtool permet d'intégrer l'incertitude dans de nombreux paramètres des modèles opérationnels grâce à la spécification facultative des distributions de probabilité. Afin d'isoler davantage les effets de certaines sources d'incertitude sur le rendement des procédures de gestion, nous élaborons d'autres modèles opérationnels qui modifient la valeur (ou la distribution) d'un ou de plusieurs paramètres ou sources de données d'intérêt (section~\ref{sec:approach3-oms}).

La pratique exemplaire recommande d'étalonner ou de conditionner les modèles opérationnels à l'aide des données observées, afin qu'ils puissent reproduire les observations historiques. Le progiciel complémentaire de DLMtool, MSEtool (Huynh et al. \protect\hyperlink{ref-huynh_msetool_2019}{2019}), comprend une mise en œuvre efficace d'une analyse de réduction du stock (Kimura and Tagart \protect\hyperlink{ref-kimura1982}{1982}; Walters et al. \protect\hyperlink{ref-walters2006}{2006}), qui est en fait un modèle statistique des prises selon l'âge qui estime les combinaisons historiques de la mortalité par pêche et du recrutement qui correspondraient aux données observées. L'analyse de réduction du stock est décrite en détail à l'annexe B de Anderson et al. (\protect\hyperlink{ref-anderson2020gfmp}{2020}\protect\hyperlink{ref-anderson2020gfmp}{a}).

Le cadre de simulation comporte deux périodes distinctes~: 1) la période historique, qui comprend toutes les années, de la première année de la série chronologique des prises observées \(t_1\) à la dernière année de cette série chronologique \(t_c\) (où « c » représente l'année « en cours »); 2) la période de projection, qui va de la première année suivant \(t_c\) à la dernière année de la projection \(t_N\). La période historique est conditionnée par des observations historiques à l'aide de l'analyse de réduction du stock (voir l'annexe B dans Anderson et al. \protect\hyperlink{ref-anderson2020gfmp}{2020}\protect\hyperlink{ref-anderson2020gfmp}{a}). Les simulations en boucle fermée, avec application des procédures de gestion et calcul des paramètres de rendement, commencent la première année de la période de projection (année \(t_{c+1}\)).

L'élaboration d'un modèle opérationnel dans le cadre des procédures de gestion comporte trois étapes.
\begin{enumerate}
\def\labelenumi{\arabic{enumi}.}
\item
  Définir les valeurs et les plages des paramètres dans le modèle opérationnel.
\item
  Envoyer les paramètres du modèle opérationnel dans le modèle d'analyse de réduction du stock, qui conditionne le modèle opérationnel en l'ajustant aux prises observées historiques, aux indices de l'abondance et à toutes les données accessibles sur la composition selon l'âge. On obtient des estimations conditionnées des paramètres du modèle et des estimations de la biomasse historique et de la mortalité historique par pêche (les années \(t_1\) à \(t_c\)), qui sont conformes aux observations historiques.
\item
  Renvoyer les valeurs des paramètres conditionnés au modèle opérationnel (maintenant le modèle opérationnel « conditionné ») pour les utiliser dans les projections de simulation en boucle fermée, à partir de l'année \(t_{c+1}\).
\end{enumerate}
Dans la mesure du possible, nous avons dérivé les paramètres du modèle opérationnel à partir de toutes les données biologiques accessibles des relevés dans la zone 4B, qui sont principalement recueillies dans le cadre des relevés à la palangre sur fond dur dans les eaux intérieures (annexe~\ref{app:biological-data}). Nous avons tiré d'autres paramètres de la documentation scientifique et des évaluations des stocks de sébaste aux yeux jaunes dans d'autres régions (voir les renseignements détaillés dans l'annexe~\ref{app:desc-om-yelloweye}). Une liste des réglages « par défaut » des paramètres du modèle opérationnel recommandés pour la plupart des stocks de poisson de fond de la Colombie-Britannique est fournie à l'annexe C de Anderson et al. (\protect\hyperlink{ref-anderson2020gfmp}{2020}\protect\hyperlink{ref-anderson2020gfmp}{a}).

Nous avons conditionné les modèles opérationnels avec l'analyse de réduction du stock, en utilisant les données sur la composition selon l'âge provenant des relevés de recherche (annexe~\ref{app:biological-data}), les indices des relevés à la palangre sur fond dur dans les eaux intérieures (annexe~\ref{app:index-data}) et les données sur les prises des pêches commerciales et récréatives (annexe~\ref{app:catch-data}). Les résultats du conditionnement des modèles opérationnels sont fournis ci-après dans la section~\ref{sec:approach3-conditioning}.

\hypertarget{sec:approach3-oms}{%
\subsection{MODÈLES OPÉRATIONNELS}\label{sec:approach3-oms}}

La pratique exemplaire en matière d'évaluation de la stratégie de gestion recommande de diviser les essais en un « ensemble de référence » de modèles opérationnels de base, qui comprend les incertitudes les plus importantes (p.~ex., épuisement du stock ou plage des valeurs de la mortalité naturelle) et un « ensemble de robustesse », afin de refléter un plus grand éventail d'incertitudes peut-être moins plausibles, mais qu'il est néanmoins intéressant d'explorer (Rademeyer et al. \protect\hyperlink{ref-rademeyer2007}{2007}). Anderson et al. (\protect\hyperlink{ref-anderson2020gfmp}{2020}\protect\hyperlink{ref-anderson2020gfmp}{a}) recommandent de présenter séparément les paramètres de rendement des ensembles de référence et de robustesse. Ils préconisent, pour la plupart des résultats, de calculer la moyenne des paramètres de rendement de l'ensemble de référence pour tous les scénarios de l'ensemble de référence de modèles opérationnels (une approche d'ensemble à intégrer pour toutes les incertitudes du modèle opérationnel), mais de présenter séparément les paramètres de rendement des différents scénarios de l'ensemble de robustesse de modèles opérationnels. La présentation distincte des résultats de l'ensemble de robustesse permet aux gestionnaires de voir comment les procédures de gestion qui ont donné de bons résultats dans l'ensemble de référence se comportent pour un ensemble d'hypothèses plus diversifiées (Rademeyer et al. \protect\hyperlink{ref-rademeyer2007}{2007}).

Pour le sébaste aux yeux jaunes des eaux intérieures, nous avons établi quatre modèles opérationnels de l'ensemble de référence~: (1) un modèle opérationnel de référence; (2) un modèle opérationnel reflétant une autre hypothèse au sujet de l'ampleur des prises historiques entre 1986 et 2005; (3) un modèle opérationnel prévoyant des événements de recrutement futurs épisodiques (rares, mais importants); (4) un modèle opérationnel estimant la sélectivité dans le relevé à la palangre sur fond dur (tableau~\ref{tab:ye-scen}).

Nous avons également établi deux modèles opérationnels de l'ensemble de robustesse englobant d'autres sources d'incertitude~: (A) un modèle opérationnel qui suppose une mortalité naturelle plus faible que les autres modèles opérationnels; (B) un modèle opérationnel qui suppose un coefficient de variation (CV) plus élevé dans le futur relevé à la palangre sur fond dur (tableau~\ref{tab:ye-scen}).
\begin{longtable}[]{@{}ll@{}}
\caption{\label{tab:ye-scen}Scénarios de modèle opérationnel pour le sébaste aux yeux jaunes des eaux intérieures.}\tabularnewline
\toprule
Nom du scénario de modèle opérationnel & Type d'ensemble\tabularnewline
\midrule
\endfirsthead
\toprule
Nom du scénario de modèle opérationnel & Type d'ensemble\tabularnewline
\midrule
\endhead
(1) Base & Reference\tabularnewline
(2) Low catch & Reference\tabularnewline
(3) Episodic recruitment & Reference\tabularnewline
(4) Estimate HBLL selectivity & Reference\tabularnewline
(A) Low M & Robustness\tabularnewline
(B) High HBLL CV & Robustness\tabularnewline
\bottomrule
\end{longtable}
\hypertarget{sec:approach3-reference}{%
\subsubsection{Ensemble de référence}\label{sec:approach3-reference}}

Les modèles opérationnels suivants ont été élaborés en tant qu'ensemble de référence. Nous les désignons ci-après par leurs numéros, p.~ex., scénario de modèle opérationnel (1).

\hypertarget{sec:approach3-reference1}{%
\subsubsection{(1) Base}\label{sec:approach3-reference1}}

Les sources de données sont fournies dans les annexes~\ref{app:biological-data} à~\ref{app:catch-data}. Les réglages des paramètres du modèle opérationnel de base sont indiqués à l'annexe~\ref{app:desc-om-yelloweye}. Nous donnons ici une brève description des hypothèses du modèle opérationnel de base qui ont été ajustées dans d'autres scénarios de modèle opérationnel.

Deux grandes incertitudes sont associées à la série chronologique des prises commerciales historiques pour le sébaste aux yeux jaunes des eaux intérieures (renseignements détaillés dans l'annexe~\ref{app:catch-data}, section~\ref{sec:com-catch-data})~: (1) les rapports sur les sébastes regroupés sous Autres sébastes (espèces de sébastes autres que le sébaste à longue mâchoire) et (2) l'ampleur des prises non déclarées qui ont été rejetées en mer avant la mise en place de la surveillance en mer de 100 \% pour la flottille de pêche à la palangre du poisson de fond en 2006 (Stanley et al. \protect\hyperlink{ref-stanley2009}{2009}). Dans un souci d'uniformité avec Yamanaka et al. (\protect\hyperlink{ref-yamanaka2011}{2011}), nous avons doublé les données sur les prises nominales pour la période 1986---2005, car l'industrie ne jugeait pas fiables les données sur les prises pour ces années {[}DFO (\protect\hyperlink{ref-dfo2012}{2012}); voir l'annexe~\ref{app:catch-data}, section~\ref{sec:com-catch-data}{]}.

Les écarts de recrutement projetés ont été échantillonnés dans l'espace logarithmique avec l'écart-type \(\tau = 0,4\), avec une autocorrélation estimée a posteriori à partir des écarts de recrutement historiques dans le modèle d'analyse de réduction du stock (annexe A de Anderson et al. \protect\hyperlink{ref-anderson2020gfmp}{2020}\protect\hyperlink{ref-anderson2020gfmp}{a}).

Le stock de sébaste aux yeux jaunes des eaux intérieures est indexé par deux relevés indépendants de la pêche~: le relevé à la palangre sur fond dur dans les eaux intérieures (annexe ({\textbf{???}})(app:index-data), section~\ref{sec:hbll-index-data}) et le relevé sur l'aiguillat commun (annexe~\ref{app:index-data}, section~\ref{sec:dogfish-index-data}). Le modèle d'analyse de réduction du stock affichait un meilleur comportement rétrospectif lorsqu'il était ajusté avec une plus grande pondération de la vraisemblance appliquée au relevé sur l'aiguillat commun. L'âge de la pleine sélectivité dans le relevé à la palangre sur fond dur a été fixé à 22 ans (voir la section~\ref{sec:approach3-conditioning} ci-après).

Les CPUE commerciales historiques étaient également accessibles et ont été utilisées comme indice de l'abondance pour le conditionnement du modèle opérationnel (Yamanaka et al. \protect\hyperlink{ref-yamanaka2011}{2011}). À la suite des décisions prises pour l'évaluation de 2011, la série chronologique a été divisée en trois phases (1986-1990, 1995-2001 et 2003-2005), représentant les périodes où le comportement de la pêche a probablement changé en réponse aux règlements de gestion (annexe~\ref{app:catch-data}, section~\ref{sec:management-changes}).

La mortalité naturelle (\emph{M}) a été échantillonnée selon une distribution de probabilité fondée sur celle utilisée par Yamanaka et al. (\protect\hyperlink{ref-yamanaka2011}{2011}), où \(M \sim \textrm{Lognormal}(0,045, 0,2)\) (annexe~\ref{app:desc-om-yelloweye}, section~\ref{app:desc-stock-m-yelloweye}).

Au cours de la période de projection, on a supposé que seul l'indice du relevé à la palangre sur fond dur était accessible pour les procédures de gestion, puisque ce relevé est mené chaque année. La projection d'un seul indice de l'abondance est compatible avec de nombreuses procédures de gestion à données limitées, qui n'utilisent qu'un seul indice de l'abondance (voir l'annexe~\ref{app:mps}). L'erreur d'observation dans les valeurs d'indice projetées a été simulée avec des écarts aléatoires par rapport à une distribution log-normale avec une moyenne de 1 et un écart-type de 0,25 d'après l'erreur d'observation estimée dans l'indice du relevé à la palangre sur fond dur.

Tous les autres scénarios du modèle opérationnel ont été ajustés à partir de ce modèle opérationnel de référence et différaient uniquement de par les ajustements des paramètres clés ou des sources de données, décrits ci-après.

\hypertarget{sec:approach3-reference2}{%
\subsubsection{(2) Prises faibles}\label{sec:approach3-reference2}}

Dans le scénario de modèle opérationnel « Prises faibles », nous testons la sensibilité du modèle à l'hypothèse de prises importantes non déclarées pour la période 1986--2005. Au lieu de doubler les données sur les prises nominales pour cette période, nous avons ajusté l'analyse de réduction du stock à ces données. Pour les autres scénarios de modèle opérationnel, nous avons utilisé les prises reconstituées jusqu'en 1985 et les prises nominales à partir de 1986.

\hypertarget{sec:approach3-reference3}{%
\subsubsection{(3) Recrutement épisodique}\label{sec:approach3-reference3}}

Les espèces longévives à maturation tardive, comme les sébastes du Pacifique, affichent souvent des stratégies de recrutement épisodiques ou périodiques caractérisées par une fécondité élevée et des épisodes occasionnels de recrutement important (Winemiller and Rose \protect\hyperlink{ref-winemiller1992}{1992}; Rose et al. \protect\hyperlink{ref-rose2001}{2001}; Winemiller \protect\hyperlink{ref-winemiller2005}{2005}). Cette stratégie du cycle biologique est parfois appelée l'effet de stockage (Warner and Chesson (\protect\hyperlink{ref-warner1985}{1985})) parce que les épisodes de fort recrutement sont stockés dans la population adulte et peuvent contribuer à la reproduction, parfois de façon significative, lorsque les conditions favorables reviennent. On pense que la longévité du sébaste a évolué comme stratégie pour résister à des conditions difficiles. Des classes d'âge très importantes ou même extrêmes ont été observées pour plusieurs espèces de sébastes de la Colombie-Britannique (p.~ex., le sébaste à longue mâchoire~: Haigh et al. \protect\hyperlink{ref-haigh2019}{2019}; le bocaccio~: Haigh and Starr \protect\hyperlink{ref-haigh2020}{2020}).

Afin de tenir compte de la possibilité que le futur recrutement durant la période de projection puisse être caractérisé par des épisodes occasionnels de recrutement très important, nous avons inclus un scénario de modèle opérationnel de « recrutement épisodique ». Ce scénario de modèle opérationnel répond à la préoccupation selon laquelle la distribution log-normale pour les écarts de recrutement qui a été utilisée dans le scénario de modèle opérationnel (1) ne modélise pas correctement les très grandes cohortes. Dans le scénario de modèle opérationnel de recrutement épisodique, les écarts de recrutement \(\varepsilon_{R,y}\) pour chaque année de la période de projection sont générés comme suit~:
\begin{equation}
\varepsilon_{R,y} = 
\left\{
\begin{array}{ll}
\varepsilon^{(1)}_{R,y} & \eta_y = 0\\
\varepsilon^{(3)}_{R,y} & \eta_y = 1,
\end{array}
\right.
\end{equation}
où \(\varepsilon^{(1)}_{R,y}\) est l'écart de recrutement par rapport au scénario de modèle opérationnel (1) et \(\log\varepsilon^{(3)}_{R,y} \sim \textrm{Normal}(-0,5\tau^2, \tau)\) représente la distribution du recrutement « épisodique » avec \(\tau = 2\) (écart-type). Le paramètre \(\eta_y\) est une variable aléatoire de Bernoulli \(\eta_y \sim \textrm{Bernoulli}(p = 1/38)\), qui sélectionne si un événement de recrutement extrême se produira. Nous supposons qu'un événement de recrutement extrême devrait se produire une fois par génération (38 ans), d'après l'observation selon laquelle des épisodes de fort recrutement chez le sébaste aux yeux jaunes des eaux intérieures ont eu lieu en 1948 et en 1970. Bien que les conditions environnementales récentes puissent être favorables à certaines espèces de sébastes (Haigh and Starr \protect\hyperlink{ref-haigh2020}{2020}; Lincandeo et al. \protect\hyperlink{ref-lincandeo2020}{2020}), nous n'avons pas encore de preuve de récents épisodes de recrutement important pour le sébaste aux yeux jaunes des eaux intérieures qui pourraient indiquer des épisodes de fort recrutement plus fréquents.

\hypertarget{sec:approach3-reference4}{%
\subsubsection{(4) Estimer la sélectivité du relevé à la palangre sur fond dur}\label{sec:approach3-reference4}}

Les tailles des échantillons annuels de données sur la composition selon l'âge provenant des relevés de recherche sont très petites (annexe~\ref{app:biological-data}). De ce fait, l'estimation de la sélectivité des relevés est très incertaine, ce qui a mené au choix de fixer la sélectivité des relevés dans les autres scénarios de modèle opérationnel.

Compte tenu de la grande incertitude entourant notre choix de sélectivité, nous avons laissé l'analyse de réduction du stock estimer la sélectivité des relevés dans ce scénario de modèle opérationnel et nous avons utilisé les données accessibles sur la composition selon l'âge dans les relevés.

\hypertarget{sec:approach3-robustness}{%
\subsubsection{Ensemble de robustesse}\label{sec:approach3-robustness}}

Les modèles opérationnels suivants ont été élaborés en tant qu'ensemble de robustesse. Nous les désignons ci-après par leurs lettres.

\hypertarget{sec:approach3-referenceA}{%
\subsubsection{(A) M faible}\label{sec:approach3-referenceA}}

Des valeurs inférieures de la mortalité naturelle ont été utilisées pour le sébaste aux yeux jaunes des eaux intérieures (Yamanaka and Lacko \protect\hyperlink{ref-yamanaka2001}{2001}; COSEWIC \protect\hyperlink{ref-cosewic2008}{2008}; Wood et al. \protect\hyperlink{ref-wood2019}{2019}). Ce scénario de modèle opérationnel utilisait une moyenne plus faible dans la distribution pour \emph{M}, avec \(M \sim \textrm{Lognormal}(0,025, 0,2)\), reflétant la possibilité que le stock soit moins productif que ce qui est supposé dans les autres scénarios de modèle opérationnel.

\hypertarget{sec:approach3-referenceB}{%
\subsubsection{(B) CV supérieur du relevé à la palangre sur fond dur}\label{sec:approach3-referenceB}}

Ce scénario de modèle opérationnel envisage la possibilité que le futur indice du relevé à la palangre sur fond dur soit moins précis que ce qui est supposé dans les autres scénarios de modèle opérationnel. Au lieu d'un écart-type de l'observation \(\sigma_I = 0,25\), nous utilisons l'écart-type (\(\sigma_I\)) et l'autocorrélation (\(\theta_\textrm{AC}\)) tirés des résidus de l'indice dans le scénario de modèle opérationnel (1), obtenus à partir de l'analyse de réduction du stock ajustée à l'indice du relevé à la palangre sur fond dur.

Cet écart-type \(\sigma_I\) a une moyenne de 0.41 et une plage de 0.38--0.44.

\hypertarget{sec:approach3-conditioning}{%
\subsection{CONDITIONNEMENT DES MODÈLES OPÉRATIONNELS}\label{sec:approach3-conditioning}}

Après avoir spécifié les paramètres des modèles opérationnels (annexe~\ref{app:desc-om-yelloweye}), nous avons conditionné les modèles opérationnels en utilisant le modèle d'analyse de réduction du stock décrit à l'annexe B de Anderson et al. (\protect\hyperlink{ref-anderson2020gfmp}{2020}\protect\hyperlink{ref-anderson2020gfmp}{a}).

Il convient de noter que le modèle opérationnel de l'outil DLMtool regroupe toutes les flottilles en une seule. Toutefois, si le modèle opérationnel est conditionné à l'aide du modèle d'analyse de réduction du stock, l'analyse de réduction du stock peut tenir compte de plusieurs flottilles et la sélectivité est propre à chaque flottille. Dans ce cas, la sélectivité de la pêche dans le modèle opérationnel pour la période de projection est remplacée par les estimations, conditionnées par l'analyse de réduction du stock, de la mortalité selon l'âge dans la pêche pour la dernière année de la période historique (\(t_c\)), normalisées en divisant par la mortalité par pêche apicale pour cette année. En gros, cela fournit au modèle opérationnel de l'outil DLMtool une sélectivité relative selon l'âge, pondérée par les prises dans toutes les flottilles. Les projections de simulation en boucle fermée supposent donc que la sélectivité relative entre les flottilles demeure constante pendant la période de projection.

De même, si le modèle opérationnel est conditionné à l'aide du modèle d'analyse de réduction du stock, les analystes peuvent également préciser (ou estimer) les paramètres de sélectivité pour les différents indices de l'abondance (dans ce cas, deux relevés indépendants de la pêche et trois séries de CPUE commerciales (figure~\ref{fig:survey-fits})). Dans ce cas, l'analyse de réduction du stock renvoie tous les indices dans l'outil DLMtool, en conservant les sélectivités selon l'âge estimées ou définies par l'utilisateur pour chaque indice. Toutefois, il convient de noter que les procédures de gestion de l'outil DLMtool n'utilisent qu'un seul indice de l'abondance (voir l'annexe~\ref{app:mps}). Dans la présente étude, toutes les procédures de gestion fondées sur des indices utilisent le relevé à la palangre sur fond dur dans les eaux intérieures.

Nous avons utilisé l'analyse de réduction du stock pour remplir les paramètres suivants dans les modèles opérationnels conditionnés~:
\begin{itemize}

\item
  \(B_{t_c}/B_0\) (ou « D »; épuisement au cours de la dernière année historique \(t_c\))
\item
  \(R_0\) (recrutement non exploité)
\item
  \(\theta_\textrm{AC}\) (ou « AC »; autocorrélation de premier ordre des écarts de recrutement)
\item
  \(\varepsilon_{\textrm{R},y}\) pour les années \(t_1\) à \(t_c\) (écarts annuels de recrutement)
\item
  \(F_{a,y}\) (mortalité par pêche selon l'âge, par année)
\end{itemize}
Des renseignements détaillés sur ces paramètres se trouvent à l'annexe B de Anderson et al. (\protect\hyperlink{ref-anderson2020gfmp}{2020}\protect\hyperlink{ref-anderson2020gfmp}{a}).

L'analyse de réduction du stock a été exécutée pour 250 répétitions. Chaque répétition utilisait une valeur différente de \emph{M} et \emph{h} (échantillonnée indépendamment des distributions indiquées à l'annexe~\ref{app:desc-om-yelloweye}, sauf pour le scénario de modèle opérationnel (A), qui utilisait une distribution différente pour \emph{M}). Le modèle a été initialisé en supposant que la biomasse féconde (\(B_y\)) était dans un état d'équilibre non exploité avant 1918, la première année de la série chronologique, c.-à-d.~\(B_{1918} = B_0\). Bien que cela ne soit probablement pas vrai, étant donné que les Premières Nations et d'autres groupes pêchaient des sébastes aux yeux jaunes avant 1918, ces nombres devraient être suffisamment faibles pour ne pas avoir d'incidence sur le rendement des procédures de gestion pendant la période de projection.

\hypertarget{sec:approach3-conditioning-base-om}{%
\subsubsection{Sélection du modèle opérationnel de base}\label{sec:approach3-conditioning-base-om}}

Les premières tentatives d'ajustement du modèle d'analyse de réduction du stock n'ont pas donné de bons ajustements au relevé sur l'aiguillat commun. De plus, l'analyse rétrospective a révélé un biais rétrospectif persistant dans les estimations annuelles de la biomasse féconde lorsque l'on supprime séquentiellement 11 années de données (lorsqu'elles sont évaluées aux valeurs moyennes de \emph{M} et \emph{h}, figure~\ref{fig:retro-initial}, graphique du haut). Le choix de fonder l'analyse rétrospective sur 11 années de données visait principalement à évaluer la sensibilité du modèle à l'élimination de chaque année de données, en remontant en 2009, la dernière année pour l'évaluation de 2011.

L'augmentation de la pondération de la vraisemblance pour l'analyse de réduction du stock ajustée au relevé sur l'aiguillat commun (\(\lambda^I_s = 4\); équation B.22 dans Anderson et al. \protect\hyperlink{ref-anderson2020gfmp}{2020}\protect\hyperlink{ref-anderson2020gfmp}{a}) et le réglage à 22 ans de l'âge à la pleine sélectivité dans le relevé à la palangre sur fond dur ont permis d'éliminer le profil rétrospectif (figure~\ref{fig:retro-initial}, graphique du bas). Nous avons donc, pour le scénario de modèle opérationnel (1) et tous les autres scénarios de modèle opérationnel à l'exception du scénario (4), fixé à 22 ans l'âge à la pleine sélectivité du relevé à la palangre sur fond dur (figure~\ref{fig:HBLL-selectivity}), tout en augmentant la pondération du relevé sur l'aiguillat commun.


\begin{figure}[htb]

{\centering \pdftooltip{\includegraphics[width=4.25in]{C:/GitHub/yelloweye-inside/mse/figures-french/retrospective-spawning-biomass}}{Figure \ref{fig:retro-initial}} 

}

\caption{Profils rétrospectifs de la biomasse féconde pour l'ajustement initial et le scénario de modèle opérationnel (1). Les lignes colorées représentent les estimations de la biomasse féconde avec \(X\) années de données supprimées, où \(X\) est indiqué dans la légende pour chaque série.}\label{fig:retro-initial}
\end{figure}
Un certain nombre de facteurs augmentent l'incertitude dans l'estimation de la sélectivité pour le relevé sur l'aiguillat commun, p.~ex., l'absence de données biologiques de ce relevé et les changements apportés aux opérations de pêche et au type de hameçon en 2004 (annexe~\ref{app:index-data}, section~\ref{sec:dogfish-index-data}). Il y a également plusieurs différences entre les relevés à la palangre sur fond dur et à la palangre sur l'aiguillat commun (annexe~\ref{app:index-data}, section~\ref{sec:dogfish-index-data}), mais pour les raisons mentionnées, nous ne pouvons pas estimer de façon fiable la sélectivité pour le relevé sur l'aiguillat commun. Par nécessité, la sélectivité dans le relevé sur l'aiguillat commun a été réglée de manière à refléter la sélectivité dans le relevé à la palangre sur fond dur, qu'elle ait été estimée (scénario du modèle opérationnel (4)) ou fixée (tous les autres scénarios de modèle opérationnel).

Les données sur la composition selon l'âge provenant de la pêche commerciale étaient accessibles pour une seule sortie de pêche en 1989, et des échantillons de longueur de 2002--2019 ont été déterminés à partir de la pêche récréative. Les tentatives visant à ajuster le modèle d'analyse de réduction du stock à ces données n'ont pas donné d'estimations satisfaisantes de la sélectivité. Les valeurs estimées de l'âge de la pleine sélectivité étaient très élevées, ce qui suggère que la plupart des âges n'étaient pas complètement vulnérables à la pêche, et elles différaient considérablement des valeurs estimées pour la population de sébaste aux yeux jaunes des eaux extérieures (Cox et al. \protect\hyperlink{ref-cox2020}{2020}). Comme il est peu probable que la sélectivité varie à ce point entre les régions géographiques, ces données n'ont pas été prises en compte pour la suite.

Par conséquent, la sélectivité de ces engins a été fixée dans tous les scénarios de modèle opérationnel (figure~\ref{fig:sra-selectivity}). Les valeurs des paramètres ont été établies de façon à ce que les courbes de la sélectivité selon l'âge se rapprochent de celles estimées par Cox et al. (\protect\hyperlink{ref-cox2020}{2020}) pour la pêche commerciale à la palangre et la pêche récréative du sébaste aux yeux jaunes des eaux extérieures (voir l'annexe~\ref{app:desc-om-yelloweye}, section~\ref{app:desc-fleet-selectivity-yelloweye}). Comme il a été mentionné ci-dessus, ces sélectivités ont été renvoyées dans les modèles opérationnels de l'outil DLMtool sous forme de courbe de sélectivité combinée des flottilles de la dernière année (\(t_c\)) de la période historique (figure~\ref{fig:om-selectivity}).

\hypertarget{sec:approach3-conditioning-results}{%
\subsubsection{Résultats du conditionnement des modèles opérationnels}\label{sec:approach3-conditioning-results}}

Les sections suivantes décrivent les résultats du conditionnement des modèles opérationnels.

\hypertarget{sec:approach3-conditioning-indices}{%
\subsubsection{Ajustements aux données}\label{sec:approach3-conditioning-indices}}

Les prises prévues dans les modèles d'analyse de réduction du stock correspondaient aux données sur les prises par leur conception, puisque l'écart-type de l'erreur d'observation était fixé à une valeur de 0,01 (Anderson et al. \protect\hyperlink{ref-anderson2020gfmp}{2020}\protect\hyperlink{ref-anderson2020gfmp}{a}, leur équation B.27).

Il a été possible d'ajuster raisonnablement bien l'analyse de réduction du stock aux indices de l'abondance (figure~\ref{fig:survey-fits}) et la convergence a été atteinte pour toutes les répétitions dans tous les scénarios de modèle opérationnel. Pour tous les scénarios de modèle opérationnel, l'indice estimé se situait dans les intervalles de confiance observés la plupart des années, malgré quelques valeurs aberrantes (figure~\ref{fig:survey-fits}). L'ajustement aux observations du début de 1986 et de 1989 dans le relevé sur l'aiguillat commun était légèrement meilleur dans le scénario de modèle opérationnel (A), correspondant à un stock moins productif et plus épuisé. Toutefois, ces deux observations sont plus incertaines que celles des dernières années, en raison des changements apportés en 2004 aux opérations de pêche et au type d'hameçon (annexe~\ref{app:index-data}, section~\ref{sec:dogfish-index-data}). Les ajustements à la série des CPUE commerciales étaient généralement bons, en raison des courtes phases pour chaque série (figure~\ref{fig:survey-fits}).


\begin{figure}[htb]

{\centering \pdftooltip{\includegraphics[width=\textwidth]{C:/GitHub/yelloweye-inside/mse/figures-french/ye-index-fits}}{Figure \ref{fig:survey-fits}} 

}

\caption{Ajustements du modèle d'analyse de réduction du stock aux indices du relevé à la palangre sur fond dur, du relevé sur l'aiguillat commun et à trois indices relatifs des CPUE commerciales. Les graphiques de gauche à droite représentent les scénarios de modèle opérationnel. Les lignes fines représentent les différents ajustements du modèle d'analyse de réduction du stock parmi les tirages stochastiques des divers paramètres du modèle opérationnel. Les points représentent la moyenne de l'indice et les segments de lignes représentent deux fois les erreurs-types entrées dans les modèles d'analyse de réduction du stock.}\label{fig:survey-fits}
\end{figure}
Le modèle d'analyse de réduction du stock est raisonnablement bien ajusté aux données sur la composition selon l'âge du relevé, malgré la très petite taille des échantillons (figures~\ref{fig:sra-conditioned-comp-fit1} à~\ref{fig:sra-conditioned-comp-fitB}). Il est à noter qu'en comparant les scénarios de modèle opérationnel (1) et (4), il semble que le réglage ou l'estimation de la sélectivité dans le relevé à la palangre sur fond dur n'a pas produit des ajustements très différents des données sur la composition selon l'âge (figures~\ref{fig:sra-conditioned-comp-fit1} et~\ref{fig:sra-conditioned-comp-fit4}).

Pour la plupart des années, l'analyse de réduction du stock a prédit une abondance plus importante dans le groupe « plus » (80+\emph{y}) que celle qui a été observée. Le manque de poissons âgés de plus de 80 \emph{y} dans le relevé permet de penser que la mortalité totale aurait pu être plus élevée dans le passé que celle qui avait été estimée dans nos modèles opérationnels. On pourrait le représenter dans les modèles opérationnels de deux façons. Premièrement, on pourrait augmenter la valeur de \emph{M} dans les modèles. Cependant, nos études préliminaires sur les modèles ont indiqué qu'une augmentation de \emph{M} accroît également le biais rétrospectif. Deuxièmement, on pourrait modifier l'historique des prises de manière à prévoir des valeurs plus élevées de la mortalité par pêche. Cela a été fait dans le scénario de modèle opérationnel (2) en supposant des prises plus faibles de 1986 à 2005, ce qui a donné un stock légèrement plus petit (figure~\ref{fig:biomass-om}) et une mortalité par pêche plus élevée (figure~\ref{fig:F-om}). Cependant, l'estimation du groupe « plus » n'a été que légèrement réduite la plupart des années pour ce scénario de modèle opérationnel (figure~\ref{fig:sra-conditioned-comp-fit2}).

Ce problème est un défi pour les espèces très longévives dont la composition selon l'âge n'est pas bien échantillonnée. Avec de si petits échantillons, les chances d'observer de très vieux poissons sont relativement faibles. En raison des grandes incertitudes structurelles du modèle et des données accessibles (p.~ex., le relevé sur l'aiguillat commun, les données sur les prises), il a été difficile de résoudre un problème, la surprédiction persistante des vieux poissons, sans en créer un autre, le biais rétrospectif.

\clearpage


\begin{figure}[htb]

{\centering \pdftooltip{\includegraphics[width=0.85\textwidth]{C:/GitHub/yelloweye-inside/mse/figures-french/conditioning/HBLL_age_comp_updog_fixsel}}{Figure \ref{fig:sra-conditioned-comp-fit1}} 

}

\caption{Ajustements du modèle d'analyse de réduction du stock aux données sur la composition selon l'âge pour le scénario de modèle opérationnel (1), montrant les proportions observées (points) et estimées (lignes vertes). La taille des échantillons (N) est le nombre d'ensembles dans lesquels des échantillons d'âge ont été prélevés chaque année.}\label{fig:sra-conditioned-comp-fit1}
\end{figure}
\clearpage


\begin{figure}[htb]

{\centering \pdftooltip{\includegraphics[width=0.85\textwidth]{C:/GitHub/yelloweye-inside/mse/figures-french/conditioning/HBLL_age_comp_lowcatch_fixsel}}{Figure \ref{fig:sra-conditioned-comp-fit2}} 

}

\caption{Ajustements du modèle d'analyse de réduction du stock aux données sur la composition selon l'âge pour le scénario de modèle opérationnel (2), montrant les proportions observées (points) et estimées (lignes vertes). La taille des échantillons (N) est le nombre d'ensembles dans lesquels des échantillons d'âge ont été prélevés chaque année.}\label{fig:sra-conditioned-comp-fit2}
\end{figure}
\clearpage

(ref:fig-sra-comp-fit3)

Ajustements du modèle d'analyse de réduction du stock aux données sur la composition selon l'âge pour le scénario de modèle opérationnel (3), montrant les proportions observées (points) et estimées (lignes vertes). La taille des échantillons (N) est le nombre d'ensembles dans lesquels des échantillons d'âge ont été prélevés chaque année.
\begin{figure}[htb]

{\centering \pdftooltip{\includegraphics[width=0.85\textwidth]{C:/GitHub/yelloweye-inside/mse/figures-french/conditioning/HBLL_age_comp_episodic_recruitment}}{Figure \ref{fig:sra-conditioned-comp-fit3}} 

}

\caption{(ref:fig-sra-comp-fit3)}\label{fig:sra-conditioned-comp-fit3}
\end{figure}
\clearpage

(ref:fig-sra-comp-fit4)

Ajustements du modèle d'analyse de réduction du stock aux données sur la composition selon l'âge pour le scénario de modèle opérationnel (4), montrant les proportions observées (points) et estimées (lignes vertes). La taille des échantillons (N) est le nombre d'ensembles dans lesquels des échantillons d'âge ont été prélevés chaque année.
\begin{figure}[htb]

{\centering \pdftooltip{\includegraphics[width=0.85\textwidth]{C:/GitHub/yelloweye-inside/mse/figures-french/conditioning/HBLL_age_comp_upweight_dogfish}}{Figure \ref{fig:sra-conditioned-comp-fit4}} 

}

\caption{(ref:fig-sra-comp-fit4)}\label{fig:sra-conditioned-comp-fit4}
\end{figure}
\clearpage

(ref:fig-sra-comp-fitA)

Ajustements du modèle d'analyse de réduction du stock aux données sur la composition selon l'âge pour le scénario de modèle opérationnel (A), montrant les proportions observées (points) et estimées (lignes vertes). La taille des échantillons (N) est le nombre d'ensembles dans lesquels des échantillons d'âge ont été prélevés chaque année.
\begin{figure}[htb]

{\centering \pdftooltip{\includegraphics[width=0.85\textwidth]{C:/GitHub/yelloweye-inside/mse/figures-french/conditioning/HBLL_age_comp_lowM_fixsel}}{Figure \ref{fig:sra-conditioned-comp-fitA}} 

}

\caption{(ref:fig-sra-comp-fitA)}\label{fig:sra-conditioned-comp-fitA}
\end{figure}
\clearpage


\begin{figure}[htb]

{\centering \pdftooltip{\includegraphics[width=0.85\textwidth]{C:/GitHub/yelloweye-inside/mse/figures-french/conditioning/HBLL_age_comp_high_index_cv}}{Figure \ref{fig:sra-conditioned-comp-fitB}} 

}

\caption{Ajustements du modèle d'analyse de réduction du stock aux données sur la composition selon l'âge pour le scénario de modèle opérationnel (B), montrant les proportions observées (points) et estimées (lignes vertes). La taille des échantillons (N) est le nombre d'ensembles dans lesquels des échantillons d'âge ont été prélevés chaque année.}\label{fig:sra-conditioned-comp-fitB}
\end{figure}
\hypertarget{sec:approach3-conditioning-parameters}{%
\subsubsection{Estimation des paramètres}\label{sec:approach3-conditioning-parameters}}

Les scénarios de référence et de robustesse des modèles opérationnels ont produit une plage de valeurs estimées des paramètres (figure~\ref{fig:sra-conditioned-parameters}). (ref:fig-sra-conditioned-parameters) Histogrammes des paramètres estimés par l'analyse de réduction du stock. AC fait référence à \(\theta_\textrm{AC}\). D désigne l'épuisement (\(B_{t_c}/B_0\)). À des fins de visualisation, les limites de l'axe \emph{R}\textsubscript{0} ont été limitées à un maximum de 490 et les limites de l'axe AC à un minimum de 0,76. Cela exclut un petit nombre de répétitions.
\begin{figure}[htb]

{\centering \pdftooltip{\includegraphics[width=0.85\textwidth]{C:/GitHub/yelloweye-inside/mse/figures-french/ye-sra-estimated}}{Figure \ref{fig:sra-conditioned-parameters}} 

}

\caption{(ref:fig-sra-conditioned-parameters)}\label{fig:sra-conditioned-parameters}
\end{figure}
Les moyennes et les coefficients de variation (CV) estimés pour les points de référence \emph{F}\textsubscript{RMD}, \emph{B}\textsubscript{RMD} et RMD sont fournis dans le tableau~\ref{tab:sra-ref-pts}. Les CV sont larges en raison de la grande gamme de valeurs échantillonnées pour \emph{M} et \emph{h} (annexe~\ref{app:desc-om-yelloweye}, sections~\ref{app:desc-stock-m-yelloweye} et~\ref{app:desc-stock-h-yelloweye}).
\begin{longtable}[]{@{}llll@{}}
\caption{\label{tab:sra-ref-pts}Points de référence estimés pour chaque scénario de modèle opérationnel. Les écarts-types entre les répétitions sont indiqués entre parenthèses. PDF = relevé à la palangre sur fond dur.}\tabularnewline
\toprule
OM Scenario & F\textsubscript{RMD} (/y) & B\textsubscript{RMD} (t) & RMD (t)\tabularnewline
\midrule
\endfirsthead
\toprule
OM Scenario & F\textsubscript{RMD} (/y) & B\textsubscript{RMD} (t) & RMD (t)\tabularnewline
\midrule
\endhead
(1) Base & 63.92 (0.64) & 0.04 (0.33) & 1362.08 (0.27)\tabularnewline
(2) Faibles prises & 53.01 (1.12) & 0.04 (0.33) & 1087.02 (0.55)\tabularnewline
(3) Recrutement épisodique & 63.92 (0.64) & 0.04 (0.33) & 1362.08 (0.27)\tabularnewline
(4) Estimation de la sélectivité du RPFD & 54.75 (0.44) & 0.04 (0.33) & 1199.70 (0.18)\tabularnewline
(A) Faible M & 34.30 (0.20) & 0.02 (0.29) & 1384.54 (0.11)\tabularnewline
(B) CV élevé du RPFD & 63.92 (0.64) & 0.04 (0.33) & 1362.08 (0.27)\tabularnewline
\bottomrule
\end{longtable}
\clearpage

\hypertarget{sec:approach3-conditioning-trajectories}{%
\subsubsection{Trajectoires historiques}\label{sec:approach3-conditioning-trajectories}}

Dans tous les scénarios de modèle opérationnel, à l'exception du scénario de modèle opérationnel (A), la médiane estimée de la biomasse féconde en 2019 est au-dessus du PRL (figure~\ref{fig:biomass-om}). Dans le scénario de modèle opérationnel (A), la médiane estimée de la biomasse féconde est inférieure au PRL la plupart des années après 2000 et a une probabilité inférieure à 50 \% d'être au-dessus du PRL en 2019 (figure~\ref{fig:ref-pt}). Les scénarios de modèle opérationnel (2) et (4) présentaient également une faible probabilité d'être inférieurs au PRL pour l'année en cours. Par conséquent, selon tous les scénarios de modèle opérationnel de l'ensemble de référence et un scénario de modèle opérationnel de l'ensemble de robustesse, on peut déjà considérer que le stock s'est rétabli au-dessus du PRL. On estime que la médiane de la biomasse féconde est supérieure au RSS dans les scénarios de modèle opérationnel (1), (2), (3) et (B) et inférieure au RSS dans les scénarios de modèle opérationnel (4) et (A).

Tous les scénarios de modèle opérationnel prévoyaient une légère augmentation de la biomasse féconde au cours de la dernière décennie de la série chronologique (figure~\ref{fig:biomass-om}). L'intervalle de crédibilité pour tous les scénarios de modèle opérationnel sauf (A) était très large, découlant des incertitudes relatives à la mortalité naturelle et au taux de variation de la pente (figure~\ref{fig:sra-conditioned-parameters}). Il convient de noter que les trajectoires sont identiques pour les scénarios de modèle opérationnel (1), (3) et (B), car les scénarios de modèle opérationnel (3) et (B) ne diffèrent du scénario de modèle opérationnel (1) que par le traitement des paramètres durant la période de projection. L'intervalle de crédibilité très étroit pour le scénario de modèle opérationnel (A) reflète la valeur fixe très faible de \emph{M} dans ce scénario de modèle opérationnel, qui limite l'éventail des résultats possibles. Les trajectoires implicites d'épuisement de la biomasse féconde durant la période historique d'après les huit modèles opérationnels suivent le même profil que celles de \emph{B} (figure~\ref{fig:depletion-om}).

Les estimations des écarts de recrutement historiques étaient semblables dans tous les scénarios de modèles opérationnels (figure~\ref{fig:recdev-om}).

La mortalité historique estimée apicale par pêche variait selon les scénarios de modèle opérationnel, avec des valeurs plus élevées estimées pour les scénarios de modèle opérationnel (4) et (A), les trajectoires avec le plus d'épuisement (figure~\ref{fig:F-om}). Tous les scénarios de modèle opérationnel prévoyaient des pics importants de la mortalité par pêche dans les années 1980 et 1990 (figure~\ref{fig:F-om}).


\begin{figure}[htb]

{\centering \pdftooltip{\includegraphics[width=\textwidth]{C:/GitHub/yelloweye-inside/mse/figures/ye-compare-SRA-MSY-panel}}{Figure \ref{fig:biomass-om}} 

}

\caption{Trajectoires de la biomasse féconde par rapport à la biomasse féconde au rendement maximal soutenu (\emph{B}/\emph{B}\textsubscript{RMD}) pour les modèles opérationnels des ensembles de référence et de robustesse. Les lignes représentent les médianes, et les ombres gris foncé et gris pâle représentent les quantiles de 50 \% et 95 \% entre les répétitions, respectivement. Les lignes horizontales en pointillés représentent le RSS (0,8\emph{B}\textsubscript{RMD}) et le PRL (0,4\emph{B}\textsubscript{RMD}).}\label{fig:biomass-om}
\end{figure}

\begin{figure}[htb]

{\centering \pdftooltip{\includegraphics[width=2.5in]{C:/GitHub/yelloweye-inside/mse/figures/historical_indicators_ref_pt}}{Figure \ref{fig:ref-pt}} 

}

\caption{Probabilité que la biomasse féconde en 2019 soit supérieure au PRL et au RSS pour les six modèles opérationnels.}\label{fig:ref-pt}
\end{figure}

\begin{figure}[htb]

{\centering \pdftooltip{\includegraphics[width=\textwidth]{C:/GitHub/yelloweye-inside/mse/figures/ye-compare-SRA-depletion-panel}}{Figure \ref{fig:depletion-om}} 

}

\caption{Trajectoires de l'épuisement de la biomasse féconde pour les modèles opérationnels des ensembles de référence et de robustesse. L'épuisement est représenté sous la forme d'une fraction de \(B_0\) (biomasse féconde à l'équilibre à un taux d'exploitation nul). Les lignes représentent les répétitions individuelles.}\label{fig:depletion-om}
\end{figure}

\begin{figure}[htb]

{\centering \pdftooltip{\includegraphics[width=\textwidth]{C:/GitHub/yelloweye-inside/mse/figures/ye-compare-SRA-recdev-panel}}{Figure \ref{fig:recdev-om}} 

}

\caption{Écarts historiques de recrutement estimés par le modèle d'analyse de réduction du stock (dans l'espace logarithmique). Les lignes représentent les répétitions individuelles.}\label{fig:recdev-om}
\end{figure}

\begin{figure}[htb]

{\centering \pdftooltip{\includegraphics[width=\textwidth]{C:/GitHub/yelloweye-inside/mse/figures/ye-compare-SRA-F-panel}}{Figure \ref{fig:F-om}} 

}

\caption{Trajectoires de la mortalité par pêche apicale (\(f_y\)) pour les modèles opérationnels des ensembles de référence et de robustesse. La mortalité par pêche apicale est la valeur maximale de \(F_y\) subie par les poissons de tout âge une année donnée. Les lignes représentent les répétitions individuelles.}\label{fig:F-om}
\end{figure}
\clearpage

\hypertarget{scuxe9nario-du-moduxe8le-opuxe9rationnel-pour-le-moduxe8le-de-production-excuxe9dentaire}{%
\subsection{SCÉNARIO DU MODÈLE OPÉRATIONNEL POUR LE MODÈLE DE PRODUCTION EXCÉDENTAIRE}\label{scuxe9nario-du-moduxe8le-opuxe9rationnel-pour-le-moduxe8le-de-production-excuxe9dentaire}}

L'évaluation du stock de 2011 pour le sébaste aux yeux jaunes des eaux intérieures a appliqué un modèle bayésien espace-état de production excédentaire, ajusté aux prises, aux CPUE commerciales et aux relevés à la palangre sur fond dur et sur l'aiguillat commun (Yamanaka et al. \protect\hyperlink{ref-yamanaka2011}{2011}). Elle a estimé une probabilité de 90 \% {[}neuf fois sur 10{]} que le stock soit inférieur au PRL, c'est-à-dire qu'il se trouve dans la zone critique. Les résultats de l'évaluation ont déclenché un plan de rétablissement (DFO \protect\hyperlink{ref-ifmp2018}{2018}), qui est mis à jour dans le présent document. Les modèles d'analyse de réduction du stock de tous les scénarios de modèle opérationnel de la présente analyse prédisent des trajectoires de la biomasse semblables à celles présentées dans Yamanaka et al. (\protect\hyperlink{ref-yamanaka2011}{2011}). Toutefois, seul le scénario de modèle opérationnel (A) estime que le stock a été inférieur au PRL dans les années 2000 (figure~\ref{fig:biomass-om}).

Les résultats du modèle divergeaient entre les deux analyses en raison des grandes différences de structure entre le modèle de production excédentaire utilisé dans Yamanaka et al. (\protect\hyperlink{ref-yamanaka2011}{2011}) et l'analyse de réduction du stock utilisée ici. Pour explorer les effets de la structure du modèle sur l'état du stock, nous avons adapté un modèle de production excédentaire aux indices de l'abondance des prises, des CPUE et indépendants de la pêche actuellement accessibles. Nous avons utilisé le modèle de production excédentaire mis en œuvre dans le progiciel R MSEtool (Huynh et al. \protect\hyperlink{ref-huynh_msetool_2019}{2019}), entièrement décrit dans l'annexe D de Anderson et al. (\protect\hyperlink{ref-anderson2020gfmp}{2020}\protect\hyperlink{ref-anderson2020gfmp}{a}), qui a été configuré pour ressembler le plus possible à l'évaluation de 2011, bien que le modèle appliqué ici ait estimé les paramètres en utilisant la vraisemblance maximale. Le modèle de production excédentaire a établi \emph{B}\textsubscript{RMD} à 50 \% de la biomasse non exploitée, la biomasse initiale en 1918 à 90 \% de la biomasse non exploitée, et a utilisé une distribution de probabilité a priori pour le taux intrinsèque de croissance de la population (\emph{r}) comme pénalité de vraisemblance. La distribution de probabilité a priori pour \emph{r} était normalement distribuée avec une moyenne de 0,068 et un écart-type de 0,03, d'après les valeurs utilisées dans Yamanaka et al. (\protect\hyperlink{ref-yamanaka2011}{2011}).

Les résultats du modèle de production excédentaire correspondaient plus étroitement à ceux de l'évaluation précédente, estimant que le stock était inférieur au PRL (figure~\ref{fig:spm-biomass}). Le modèle a estimé les points de référence \emph{B}\textsubscript{RMD} = 1 844 t, \emph{F}\textsubscript{RMD} = 0,03 \emph{y}\textsuperscript{-1} et RMD = 62 tonnes. Les estimations de RMD et de \emph{F}\textsubscript{RMD} étaient semblables à celles du modèle opérationnel de base de l'analyse de réduction du stock (tableau~\ref{tab:sra-ref-pts}), mais l'estimation de \emph{B}\textsubscript{RMD} était environ 35 \% plus élevée pour le modèle de production excédentaire que pour l'analyse de réduction du stock. Nous notons que Cox et al. (\protect\hyperlink{ref-cox2020}{2020}) ont tiré des conclusions très similaires dans le plan de rétablissement du stock de sébaste aux yeux jaunes des eaux extérieures. Leur analyse, également fondée sur des modèles structurés selon l'âge, a prédit que l'état actuel du stock de sébaste aux yeux jaunes des eaux extérieures se situerait au-dessus du PRL, c'est-à-dire qu'un rétablissement n'était pas nécessaire, contrairement aux résultats obtenus par le modèle de production excédentaire utilisé dans l'évaluation des stocks de 2014 (Yamanaka et al. \protect\hyperlink{ref-yamanaka2018yelloweyeoutside}{2018}).

En particulier, les tendances de la biomasse pour la population des eaux intérieures depuis 2000 diffèrent également entre les deux types de modèles. Une tendance aplatie et stable (inférieure au PRL) a été estimée dans le modèle de production excédentaire. Par ailleurs, bon nombre des modèles opérationnels élaborés à partir de l'analyse de réduction du stock indiquent que la taille du stock a augmenté, bien que la tendance soit beaucoup plus aplatie dans le scénario de M faible. Nous avons essayé quatre modèles de rechange pour tenter de trouver des scénarios qui refléteraient la tendance de la biomasse estimée dans le modèle de production excédentaire (figure~\ref{fig:alt-SRA-fit}).

Tout d'abord, le relevé sur l'aiguillat commun a été encore moins pondéré, avec un facteur de pondération de la vraisemblance \(\lambda = 0,1\). Cela n'a pas changé la tendance de la biomasse féconde et l'état du stock obtenu était plus optimiste que le modèle opérationnel de base. Ensuite, la valeur du relevé sur l'aiguillat commun de 2019 a été exclue de la vraisemblance puisque la moyenne observée est plus élevée que les dernières années (avec \(\lambda = 4\) d'après le modèle opérationnel de base). Ce scénario n'a pas non plus eu d'effet appréciable sur la tendance et l'ampleur de la biomasse féconde. Puis, les compositions selon l'âge du relevé ont été pondérées à la baisse, avec \(\lambda = 1\). Comme dans la première option, la tendance de la biomasse féconde était semblable à celle du modèle opérationnel de base, mais l'état du stock obtenu était un peu moins optimiste. Enfin, les compositions selon l'âge du relevé ont été supprimées de la vraisemblance avec \(\lambda = 0\). Ce scénario a estimé que le stock était inférieur au PRL conformément au modèle de production excédentaire.

Ces résultats permettent de penser que les différences dans l'état estimé du stock entre l'analyse actuelle et l'évaluation du stock précédente (Yamanaka et al. \protect\hyperlink{ref-yamanaka2011}{2011}) sont en grande partie attribuables aux types de données inclus dans les modèles respectifs et leur structure. Les deux utilisent les prises et les indices de l'abondance, bien que le modèle de production excédentaire ne tienne pas compte des compositions selon l'âge. Lorsque les deux modèles utilisent les mêmes types de données, c.-à-d.~en excluant les compositions selon l'âge de la vraisemblance avec des hypothèses fixes concernant la sélectivité du relevé, les deux se comportent de façon plus similaire.

Nous avons choisi le modèle structuré selon l'âge plutôt que le modèle de production excédentaire pour élaborer des modèles opérationnels pour le sébaste aux yeux jaunes des eaux intérieures selon les premiers principes. Le modèle structuré selon l'âge est plus réaliste pour modéliser les retards dans la productivité du stock au fil du temps. Une seule cohorte contribue à la biomasse féconde (et à la biomasse vulnérable) sur plusieurs années à mesure qu'elle progresse dans la structure selon l'âge de la population. Ce mécanisme peut expliquer l'augmentation de la biomasse estimée dans l'analyse de réduction du stock. À mesure que la mortalité par pêche diminuait à la suite de la réduction des prises à la fin des années 1990 jusqu'en 2000, la biomasse féconde a commencé à augmenter. De plus, l'âge à 5 \% de maturité est inférieur à l'âge à 5 \% de sélectivité de la flottille commerciale, ce qui permettrait à certaines parties des cohortes de frayer avant une vulnérabilité importante à la pêche.

Pour le modèle de production excédentaire, ce retard peut être implicitement intégré à la fonction de production du modèle, mais la biomasse prévue pour une année donnée est explicitement une fonction de la biomasse observée l'année précédente. De plus, la biomasse du stock dans le modèle de production excédentaire est implicitement la biomasse vulnérable puisqu'il n'y a pas d'hypothèses explicites sur la sélectivité. Il est plus difficile d'expliquer la productivité du stock de manière mécaniste, car tous les processus biologiques concernant la croissance, la mortalité naturelle et la maturité sont intégrés dans le paramètre de taux intrinsèque \emph{r}.


\begin{figure}[htb]

{\centering \pdftooltip{\includegraphics[width=0.8\textwidth]{C:/GitHub/yelloweye-inside/mse/figures/SP_fit}}{Figure \ref{fig:spm-biomass}} 

}

\caption{Trajectoire de la biomasse par rapport à la biomasse au rendement maximal soutenu (\emph{B}/\emph{B}\textsubscript{RMD}) dans le modèle de production excédentaire. Les lignes horizontales en pointillés représentent le RSS (0,8\emph{B}\textsubscript{RMD}) et le PRL (0,4\emph{B}\textsubscript{RMD}).}\label{fig:spm-biomass}
\end{figure}

\begin{figure}[htb]

{\centering \pdftooltip{\includegraphics[width=5.5in]{C:/GitHub/yelloweye-inside/mse/figures/alt_SRA_fit}}{Figure \ref{fig:alt-SRA-fit}} 

}

\caption{Biomasse féconde relative (\emph{B}/\emph{B}\textsubscript{RMD}) selon d'autres ajustements à l'analyse de réduction du stock qui réduisaient ou éliminaient les données de la vraisemblance. On a utilisé les valeurs moyennes de la mortalité naturelle et du taux de variation de la pente. Les lignes horizontales en pointillés représentent le RSS (0,8\emph{B}\textsubscript{RMD}) et le PRL (0,4\emph{B}\textsubscript{RMD}). Les scénarios de référence et d'exclusion du relevé de 2019 sur l'aiguillat commun se chevauchent et les lignes correspondantes sont différenciées sur la figure pour plus de clarté.}\label{fig:alt-SRA-fit}
\end{figure}
\clearpage

\hypertarget{computational-environment}{%
\section{COMPUTATIONAL ENVIRONMENT}\label{computational-environment}}

This version of the document was generated on 2021-10-26 15:05:22 with R version 3.6.3 (2020-02-29) (R Core Team \protect\hyperlink{ref-r2019}{2019}) and R package versions:
\begin{longtable}[]{@{}lll@{}}
\toprule
Package & Version & Date\tabularnewline
\midrule
\endhead
bookdown & 0.17 & 2020-01-11\tabularnewline
cowplot & 1.0.0 & 2019-07-11\tabularnewline
csasdown & 0.0.10.9000 & 2021-05-25\tabularnewline
DLMtool & 5.4.1 & 2019-12-06\tabularnewline
dplyr & 0.8.4 & 2020-01-31\tabularnewline
gfdata & 0.0.0.9000 & 2020-03-04\tabularnewline
gfdlm & 0.0.1.9000 & 2020-03-26\tabularnewline
gfplot & 0.1.4 & 2019-12-10\tabularnewline
ggplot2 & 3.2.1 & 2019-08-10\tabularnewline
knitr & 1.28 & 2020-02-06\tabularnewline
MSEtool & 1.4.3 & 2020-01-10\tabularnewline
purrr & 0.3.3 & 2019-10-18\tabularnewline
rmarkdown & 2.1 & 2020-01-20\tabularnewline
tidyr & 1.0.2 & 2020-01-24\tabularnewline
TMB & 1.7.16 & 2020-01-15\tabularnewline
\bottomrule
\end{longtable}
The source code for this document is available at:\\
\url{https://github.com/pbs-assess/yelloweye-inside/tree/2f9a8a4}.

This document was compiled with the R package csasdown (Anderson et al. \protect\hyperlink{ref-csasdown}{2020}\protect\hyperlink{ref-csasdown}{b}).

The specific versions of the primary packages used to generate this report can be viewed at:

\url{https://github.com/DLMtool/DLMtool/tree/fa971cf}\\
\url{https://github.com/tcarruth/MSEtool/tree/fa1498c}~\\
\url{https://github.com/pbs-assess/gfdata/tree/7292039}~\\
\url{https://github.com/pbs-assess/gfplot/tree/e0b36c0}~\\
\url{https://github.com/pbs-assess/gfdlm/tree/b895686}~\\
\url{https://github.com/pbs-assess/csasdown/tree/f9d5081}~\\

\vspace{4mm}

or installed via:

\texttt{\#\ install.packages(\textquotesingle{}devtools\textquotesingle{})}\\
\texttt{devtools::install\_github(\textquotesingle{}DLMtool/DLMtool\textquotesingle{},\ ref\ =\ \textquotesingle{}fa971cf\textquotesingle{})}~\\
\texttt{devtools::install\_github(\textquotesingle{}tcarruth/MSEtool\textquotesingle{},\ ref\ =\ \textquotesingle{}fa1498c\textquotesingle{})}~\\
\texttt{devtools::install\_github(\textquotesingle{}pbs-assess/gfdata\textquotesingle{},\ ref\ =\ \textquotesingle{}7292039\textquotesingle{})}~\\
\texttt{devtools::install\_github(\textquotesingle{}pbs-assess/sha\_gfplot\textquotesingle{},\ ref\ =\ \textquotesingle{}e0b36c0\textquotesingle{})}~\\
\texttt{devtools::install\_github(\textquotesingle{}pbs-assess/gfdlm\textquotesingle{},\ ref\ =\ \textquotesingle{}b895686\textquotesingle{})}~\\
\texttt{devtools::install\_github(\textquotesingle{}pbs-assess/csasdown\textquotesingle{},\ ref\ =\ \textquotesingle{}f9d5081\textquotesingle{})}~\\

\clearpage

\hypertarget{refs}{}
\leavevmode\hypertarget{ref-anderson2020gfmp}{}%
Anderson, S.C., Forrest, R.E., Huynh, Q.C., and Keppel, E.A. 2020a. A management procedure framework for groundfish in British Columbia. DFO Can. Sci. Advis. Sec. Res. Doc. 2020/nnn: In prep.

\leavevmode\hypertarget{ref-csasdown}{}%
Anderson, S.C., Grandin, C., Edwards, A.M., Grinnell, M.H., Ricard, D., and Haigh, R. 2020b. csasdown: Reproducible CSAS reports with bookdown. R package version 0.0.8. \url{https://github.com/pbs-assess/csasdown}.

\leavevmode\hypertarget{ref-carruthers2018}{}%
Carruthers, T.R., and Hordyk, A. 2018. The data-limited methods toolkit (DLMtool): An R package for informing management of data-limited populations. Methods Ecol. Evol. 9: 2388--2395.

\leavevmode\hypertarget{ref-cosewic2008}{}%
COSEWIC. 2008. COSEWIC assessment and status report on the Yelloweye Rockfish (\emph{Sebastes ruberrimus}), Pacific Ocean inside waters population and Pacific Ocean outside waters population, in Canada. Committee on the Status of Endangered Wildlife in Canada \url{https://www.sararegistry.gc.ca/virtual_sara/files/cosewic/sr_yelloweye_rockfish_0809_e.pdf}.

\leavevmode\hypertarget{ref-cox2020}{}%
Cox, S.P., Doherty, B., Benson, A.J., Johnson, S.D., and Haggarty, D. 2020. Evaluation of potential rebuilding strategies for Outside Yelloweye Rockfish in British Columbia. DFO Can. Sci. Advis. Sec. Res. Doc. 2020/nnn.

\leavevmode\hypertarget{ref-dfo2012}{}%
DFO. 2012. Survey of recreational fishing in canada 2010. DFO Res. Manage. Eco. Fish. Manage. 2012-1804.

\leavevmode\hypertarget{ref-ifmp2018}{}%
DFO. 2018. Pacific Region integrated fisheries management plan, groundfish, effective February 21, 2018 \url{http://waves-vagues.dfo-mpo.gc.ca/Library/40657814.pdf}.

\leavevmode\hypertarget{ref-haigh2020}{}%
Haigh, R., and Starr, P.J. 2020. Bocaccio (\emph{Sebastes paucispinis}) stock assessment for British Columbia in 2019, including guidance for rebuilding plans. DFO Can. Sci. Advis. Sec. Res. Doc. 2020/nnn.

\leavevmode\hypertarget{ref-haigh2019}{}%
Haigh, R., Starr, P.J., Edwards, A.M., King, J.R., and Lecomte, J.-B. 2019. Stock assessment for Pacific Ocean Perch (\emph{Sebastes alutus}) in Queen Charlotte Sound, British Columbia in 2017. DFO Can. Sci. Advis. Sec. Res. Doc. 2019/038. v + 227 pp.

\leavevmode\hypertarget{ref-huynh_msetool_2019}{}%
Huynh, Q.C., Hordyk, A.R., and Carruthers, T. 2019. MSEtool: Management strategy evaluation toolkit. R package version 1.4.3.

\leavevmode\hypertarget{ref-kimura1982}{}%
Kimura, D.K., and Tagart, J.V. 1982. Stock Reduction Analysis, another solution to the catch equations. Can. J. Fish. Aquat. Sci. 39(11): 1467--1472.

\leavevmode\hypertarget{ref-lincandeo2020}{}%
Lincandeo, R., Duplisea, D.E., Senay, C., Marantette, J.R., and McAllister, M.K. 2020. Management strategies for spasmodic stocks: A Canadian Atlantic redfish fishery case study. Can. J. Fish. Aquat. Sci. 77: 684--702.

\leavevmode\hypertarget{ref-rademeyer2007}{}%
Rademeyer, R.A., Plagányi, É.E., and Butterworth, D.S. 2007. Tips and tricks in designing management procedures. ICES J. Mar. Sci. 64(4): 618--625.

\leavevmode\hypertarget{ref-r2019}{}%
R Core Team. 2019. R: A language and environment for statistical computing. R Foundation for Statistical Computing, Vienna, Austria.

\leavevmode\hypertarget{ref-rose2001}{}%
Rose, K.A., Cowan, J.H., Winemiller, K.O., Myers, R.A., and Hilborn, R. 2001. Compensatory density dependence in fish populations: Importance, controversy, understanding and prognosis: Compensation in fish populations. Fish. Fish. 2(4): 293--327.

\leavevmode\hypertarget{ref-stanley2009}{}%
Stanley, R.D., Olsen, N., and Fedoruk, A. 2009. Independent validation of the accuracy of Yelloweye Rockfish catch estimates from the Canadian groundfish integration pilot project. Mar. Coast. Fish. 1(1): 354--362.

\leavevmode\hypertarget{ref-walters2006}{}%
Walters, C.J., Martell, S.J.D., and Korman, J. 2006. A stochastic approach to stock reduction analysis. Can. J. Fish. Aquat. Sci. 63(1): 212--223.

\leavevmode\hypertarget{ref-warner1985}{}%
Warner, R.R., and Chesson, P.L. 1985. Coexistence mediated by recruitment fluctuations: A field guide to the storage effect. The American Naturalist 125(6): 769--787.

\leavevmode\hypertarget{ref-winemiller2005}{}%
Winemiller, K.O. 2005. Life history strategies, population regulation, and implications for fisheries management. Can. J. Fish. Aquat. Sci. 62(4): 872--885.

\leavevmode\hypertarget{ref-winemiller1992}{}%
Winemiller, K.O., and Rose, K.A. 1992. Patterns of life-history diversification in North American fishes: Implications for population regulation. Can. J. Fish. Aquat. Sci. 49(10): 2196--2218.

\leavevmode\hypertarget{ref-wood2019}{}%
Wood, K., Olson, A., Williams, B., and Jaenicke, M. 2019. 14: Assessment of the demersal shelf rockfish stock complex in the southeast outside subdistrict of the gulf of alaska. NPFMC.

\leavevmode\hypertarget{ref-yamanaka2001}{}%
Yamanaka, K.L., and Lacko, L. 2001. Inshore rockfish (\emph{Sebastes ruberrimus, S. maliger, S. caurinus, S. melanops, S. nigrocinctus and S. nebulosus}) stock assessment for the west coast of Canada and recommendations for management. DFO Can. Sci. Advis. Sec. Res. Doc. 2001/139.

\leavevmode\hypertarget{ref-yamanaka2011}{}%
Yamanaka, K.L., McAllister, M.K., Olesiuk, P.F., Etienne, M.-P., Obradovich, S.G., and Haigh, R. 2011. Stock assessment for the inside population of Yelloweye Rockfish (\emph{Sebastes ruberrimus}) for British Columbia, Canada for 2010. DFO Can. Sci. Advis. Sec. Res. Doc. 2011/129. xiv + 131 p.

\leavevmode\hypertarget{ref-yamanaka2018yelloweyeoutside}{}%
Yamanaka, K.L., McAllister, M.M., Etienne, M.-P., Edwards, A.M., and Rowan Haigh. 2018. Stock assessment for the outside population of Yelloweye Rockfish (\emph{Sebastes ruberrimus}) for British Columbia, Canada in 2014. DFO Can. Sci. Advis. Sec. Res. Doc. 2018/001. ix + 150 p.

\end{document}
