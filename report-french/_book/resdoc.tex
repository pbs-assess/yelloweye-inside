% Documents setup
\documentclass[11pt]{book}

% fix for pandoc 1.14
\providecommand{\tightlist}{%
  \setlength{\itemsep}{0pt}\setlength{\parskip}{0pt}}

\usepackage{tabu} % https://tex.stackexchange.com/questions/50332/vertical-spacing-of-a-table-cell

% Location of the csas-style repository: adjust path as needed
\newcommand{\locRepo}{csas-style}

% Use the style file in the csas-style repository (res-doc.sty)
\usepackage{\locRepo/res-doc}

% header-includes from R markdown entry


% Headers and footers
\lhead{}
% \lhead{}
\rhead{}
% \rfoot{DRAFT - DO NOT CITE}

%%%% Commands for title page etc %%%%%

% Publication year
\newcommand{\rdYear}{2021}

% Publication month
\newcommand{\rdMonth}{}

% Report number
\newcommand{\rdNumber}{8}

% Region
\newcommand{\rdRegion}{Pacific Region}

% Title
\newcommand{\rdTitle}{Évaluation des stratégies de rétablissement possibles pour le sébaste aux yeux jaunes (\emph{Sebastes ruberrimus}) des eaux intérieures de la Colombie-Britannique}

\newcommand{\rdISBN}{Fs70-5/2021-008F-PDF}
\newcommand{\rdCatNo}{978-0-660-38699-7}

% Author names separated by commas and ', and' for the last author in the format 'M.H. Grinnell' (use \textsuperscript{n} for addresses)
\newcommand{\rdAuth}{Dana R. Haggarty\textsuperscript{1}, Quang C. Huynh\textsuperscript{2}, Robyn E. Forrest\textsuperscript{1}, Sean C. Anderson\textsuperscript{1}, Midoli J. Bresch\textsuperscript{1}, Elise A. Keppel\textsuperscript{1}}

% Author names reversed separated by commas in the format 'Grinnell, M.H.'
\newcommand{\rdAuthRev}{Haggarty, D.R., C.R. Huynh, R.E. Forrest, S.C. Anderson, M.J. Bresch, et E.A. Keppel}

% Author addresses (use \textsuperscript{n})
\newcommand{\rdAuthAddy}{\textsuperscript{1}Station biologique du Pacifique\\
Pêches et Océans Canada, 3190, chemin Hammond Bay\\
Nanaimo (Colombie-Britannique) V9T 6N7, Canada\\
\textsuperscript{2}Institut pour les océans et la pêche\\
LRAE de l'Université de la Colombie- Britannique, 2202, Main Mall\\
Vancouver (Colombie-Britannique) V6T 1Z4, Canada\\}

\newcommand{\citationOtherLanguage}{Haggarty, D.R., Huynh, Q.C., Forrest, R.E., Anderson, S.C., Bresch, M.J., Keppel, E.A. 2021. Evaluation of potential rebuilding strategies for Inside Yelloweye Rockfish (\emph{Sebastes ruberrimus}) in British Columbia. DFO Can. Sci. Advis. Sec. Res. Doc. 2021/008. vi + 141 p.}

% Name of file with abstract and resume (see \abstract and \frenchabstract for requirements)
\newcommand{\rdAbstract}{\abstract{En vertu des politiques et de la législation canadiennes, il faut Eétablir un plan de rétablissement pour les stocks de poissons qui ont été évalués comme étant inférieurs au point de référence limite (PRL) afin de les ramener au-delà du PRL. Les plans de rétablissement doivent être fondés sur des objectifs caractérisés par 1) une cible, 2) un délai souhaité pour atteindre la cible et 3) une probabilité acceptable d'atteindre la cible. Les plans de rétablissement doivent également comprendre des mesures de gestion ou des procédures de gestion, des jalons cibles et être évalués régulièrement. \vspace{1.5mm} \break Le stock de sébaste aux yeux jaunes (\emph{Sebastes ruberrimus}) des eaux intérieures est un stock sur lequel on dispose de données limitées, présent dans la zone de gestion du poisson de fond 4B (détroit de la Reine-Charlotte, détroit de Georgie et détroit de Juan de Fuca) en Colombie-Britannique. Il a été évalué comme étant inférieur au PLR en 2010, ce qui a donné lieu à la publication d'un plan de rétablissement. Il est également inscrit en vertu de la \emph{Loi sur les espèces en péril} comme espèce préoccupante. L'actuelle procédure de gestion pour assurer le rétablissement est un total autorisé des captures (TAC) annuel fixe de 15 tonnes métriques, qui n'a pas été réévalué depuis la dernière évaluation. \vspace{1.5mm} \break Ce projet vise à fournir un avis scientifique à l'appui de la réévaluation du plan de rétablissement du sébaste aux yeux jaunes des eaux intérieures. Nous appliquons un nouveau cadre d'évaluation de la stratégie de gestion (le Cadre des procédures de gestion), récemment élaboré pour le poisson de fond de la Colombie-Britannique, afin d'évaluer le rendement des autres procédures de gestion à données limitées pour ce qui est de l'atteinte des objectifs de rétablissement. Le Cadre des procédures de gestion suit six étapes de pratiques exemplaires pour évaluer la stratégie de gestion~: 1) la définition du contexte décisionnel; 2) l'établissement des objectifs et des paramètres de rendement; 3) la précision des modèles opérationnels pour représenter le système sous-jacent et calculer les paramètres de rendement; 4) la sélection des procédures de gestion possibles; 5) la réalisation de simulations en boucle fermée afin d'évaluer le rendement des procédures de gestion; 6) la présentation des résultats pour faciliter l'évaluation des compromis. \vspace{1.5mm} \break Nous avons appliqué ce cadre pour évaluer le rendement de 34 procédures de gestion à données limitées pour ce qui est de l'atteinte de l'objectif principal, qui est de ramener le stock au-dessus du PRL sur 1,5 génération avec au moins une probabilité de réussite de 95 \% {[}19 fois sur 20{]}. Nous avons également évalué le rendement des procédures de gestion en ce qui concerne deux autres paramètres de conservation, quatre objectifs de prises moyennes et un objectif de variabilité des prises. Pour tenir compte de l'incertitude liée à la dynamique de la population sous-jacente et aux sources de données, nous avons élaboré six scénarios de modèles opérationnels de rechange, qui différaient de par les hypothèses précises du modèle et des données. Ces scénarios de modèles opérationnels ont été divisés en un « ensemble de référence » (quatre modèles opérationnels) et un « ensemble de robustesse » (deux modèles opérationnels). Nous avons conditionné tous les modèles opérationnels aux données sur les prises observées, aux indices de l'abondance et aux données accessibles sur la composition selon l'âge. Nous avons utilisé la simulation en boucle fermée pour évaluer le rendement des procédures de gestion et nous avons éliminé celles qui ne satisfaisaient pas à un ensemble de critères de base, ce qui a laissé cinq procédures de gestion possibles~: des procédures de gestion à prises constantes annuelles de 10 ou 15 tonnes et trois procédures de gestion qui ajustent le TAC en fonction de la pente relative de l'indice de l'abondance dans le relevé à la palangre sur fond dur dans les eaux intérieures. \vspace{1.5mm} \break Les cinq procédures de gestion finales atteignaient l'objectif principal avec une probabilité supérieure à 0,98 (49 fois sur 50), dans les scénarios des quatre modèles opérationnels de l'ensemble de référence, surtout qu'aucun des modèles opérationnels de l'ensemble de référence n'a estimé que le stock serait inférieur au PRL en 2020. Dans les scénarios des deux modèles opérationnels de l'ensemble de robustesse, le scénario qui simulait une plus grande variabilité dans le futur relevé à la palangre sur fond dur a donné des résultats semblables à ceux des scénarios de l'ensemble de référence. Cependant, dans le scénario qui supposait un taux de mortalité naturelle plus faible pour le stock (« M faible »), toutes les procédures de gestion avaient des probabilités plus basses d'atteindre l'objectif principal, la probabilité la plus faible étant atteinte par la procédure de gestion actuelle (prises constantes de 15 tonnes). \vspace{1.5mm} \break Nous présentons un certain nombre de visualisations pour illustrer les compromis entre les objectifs de conservation et de prises pour les différentes procédures de gestion dans d'autres scénarios de modèles opérationnels. Ces visualisations présentent les compromis sous forme de tableaux et de graphiques, destinés à faciliter le processus de sélection de la procédure de gestion finale. Étant donné que toutes les procédures de gestion ont atteint l'objectif principal dans les scénarios de l'ensemble de référence, il n'y avait pas de compromis important entre les objectifs de conservation et les objectifs de prises. Parmi les deux scénarios de l'ensemble de robustesse, les compromis étaient les plus évidents dans le scénario de M faible, où la probabilité d'atteindre l'objectif principal diminuait à mesure que la probabilité de prises moyennes à court terme de 10 tonnes augmentait. \vspace{1.5mm} \break Nous discutons des incertitudes majeures, y compris l'incertitude entourant la mortalité naturelle, la sélectivité et les prises historiques, en notant que nous avons tenté d'en tenir compte en évaluant le rendement des procédures de gestion dans plusieurs modèles opérationnels. Nous soulignons les problèmes concernant les estimations de l'état actuel du stock de sébaste aux yeux jaunes des eaux intérieures et le rôle des points de référence dans le Cadre des procédures de gestion. Nous formulons des recommandations sur la fréquence des évaluations et suggérons des déclencheurs pour la réévaluation. Nous évaluons également le rendement des procédures de gestion en ce qui concerne le respect de deux autres critères d'évaluation pour le Comité sur la situation des espèces en péril au Canada.}}

%%%% End of title page commands %%%%%

% \pdfcompresslevel=5 % faster PNGs

\setcounter{section}{0}

\bibliographystyle{csas-style/res-doc}

\usepackage{amsmath}
\usepackage{bm}

% commands and environments needed by pandoc snippets
% extracted from the output of `pandoc -s`
%% Make R markdown code chunks work
\usepackage{array}
\usepackage{amssymb,amsmath}
\usepackage{color}
\usepackage{fancyvrb}

% From default template:
\newcommand{\VerbBar}{|}
\newcommand{\VERB}{\Verb[commandchars=\\\{\}]}
\DefineVerbatimEnvironment{Highlighting}{Verbatim}{commandchars=\\\{\}}
% Add ',fontsize=\small' for more characters per line
\usepackage{framed}
\definecolor{shadecolor}{RGB}{248,248,248}
\newenvironment{Shaded}{\begin{snugshade}}{\end{snugshade}}
\newcommand{\AlertTok}[1]{\textcolor[rgb]{0.94,0.16,0.16}{#1}}
\newcommand{\AnnotationTok}[1]{\textcolor[rgb]{0.56,0.35,0.01}{\textbf{\textit{#1}}}}
\newcommand{\AttributeTok}[1]{\textcolor[rgb]{0.77,0.63,0.00}{#1}}
\newcommand{\BaseNTok}[1]{\textcolor[rgb]{0.00,0.00,0.81}{#1}}
\newcommand{\BuiltInTok}[1]{#1}
\newcommand{\CharTok}[1]{\textcolor[rgb]{0.31,0.60,0.02}{#1}}
\newcommand{\CommentTok}[1]{\textcolor[rgb]{0.56,0.35,0.01}{\textit{#1}}}
\newcommand{\CommentVarTok}[1]{\textcolor[rgb]{0.56,0.35,0.01}{\textbf{\textit{#1}}}}
\newcommand{\ConstantTok}[1]{\textcolor[rgb]{0.00,0.00,0.00}{#1}}
\newcommand{\ControlFlowTok}[1]{\textcolor[rgb]{0.13,0.29,0.53}{\textbf{#1}}}
\newcommand{\DataTypeTok}[1]{\textcolor[rgb]{0.13,0.29,0.53}{#1}}
\newcommand{\DecValTok}[1]{\textcolor[rgb]{0.00,0.00,0.81}{#1}}
\newcommand{\DocumentationTok}[1]{\textcolor[rgb]{0.56,0.35,0.01}{\textbf{\textit{#1}}}}
\newcommand{\ErrorTok}[1]{\textcolor[rgb]{0.64,0.00,0.00}{\textbf{#1}}}
\newcommand{\ExtensionTok}[1]{#1}
\newcommand{\FloatTok}[1]{\textcolor[rgb]{0.00,0.00,0.81}{#1}}
\newcommand{\FunctionTok}[1]{\textcolor[rgb]{0.00,0.00,0.00}{#1}}
\newcommand{\ImportTok}[1]{#1}
\newcommand{\InformationTok}[1]{\textcolor[rgb]{0.56,0.35,0.01}{\textbf{\textit{#1}}}}
\newcommand{\KeywordTok}[1]{\textcolor[rgb]{0.13,0.29,0.53}{\textbf{#1}}}
\newcommand{\NormalTok}[1]{#1}
\newcommand{\OperatorTok}[1]{\textcolor[rgb]{0.81,0.36,0.00}{\textbf{#1}}}
\newcommand{\OtherTok}[1]{\textcolor[rgb]{0.56,0.35,0.01}{#1}}
\newcommand{\PreprocessorTok}[1]{\textcolor[rgb]{0.56,0.35,0.01}{\textit{#1}}}
\newcommand{\RegionMarkerTok}[1]{#1}
\newcommand{\SpecialCharTok}[1]{\textcolor[rgb]{0.00,0.00,0.00}{#1}}
\newcommand{\SpecialStringTok}[1]{\textcolor[rgb]{0.31,0.60,0.02}{#1}}
\newcommand{\StringTok}[1]{\textcolor[rgb]{0.31,0.60,0.02}{#1}}
\newcommand{\VariableTok}[1]{\textcolor[rgb]{0.00,0.00,0.00}{#1}}
\newcommand{\VerbatimStringTok}[1]{\textcolor[rgb]{0.31,0.60,0.02}{#1}}
\newcommand{\WarningTok}[1]{\textcolor[rgb]{0.56,0.35,0.01}{\textbf{\textit{#1}}}}

\newcommand{\lt}{\ensuremath <}
\newcommand{\gt}{\ensuremath >}

%Defines cslreferences environment
%Required by pandoc 2.8
%Copied from https://github.com/rstudio/rmarkdown/issues/1649

\DeclareGraphicsExtensions{.png,.pdf}
\begin{document}

\frontmatter

\section{DONNÉES SUR LES PRISES}
\label{app:catch-data}

Le sébaste aux yeux jaunes des eaux intérieures est ciblé dans les pêches commerciales à la ligne et à l'hameçon, les pêches à des fins alimentaires, sociales et rituelles (ASR) et les pêches récréatives. La gestion de la pêche du sébaste aux yeux jaunes des eaux intérieures a commencé en 1986, avec la mise en place du permis commercial de catégorie « ZN » et des limites de prises quotidiennes pour les pêcheurs récréatifs. Une chronologie des changements de gestion pour les pêches commerciales et récréatives est présentée dans les tableaux~\ref{tab:comm-mgt-changes} et~\ref{tab:rec-mgt-changes}.

\hypertarget{sec:com-catch-data}{%
\subsection{DONNÉES SUR LES PRISES COMMERCIALES}\label{sec:com-catch-data}}

Les données sur les prises de sébaste peuvent être regroupées en trois périodes~: historique (de 1918 à 1950), début de la période électronique (de 1951 à 2005) et moderne (2006 et après). Il y a deux grandes sources d'incertitude dans la période historique et le début de la période électronique pour le sébaste aux yeux jaunes des eaux intérieures. La première est que les prises de sébastes, autres que de sébaste à longue mâchoire (\emph{Sebastes alutus}), étaient déclarées de façon regroupée (autre sébaste, ORF) pendant la période historique. Pour reconstituer les prises historiques, des auteurs du MPO ont élaboré un algorithme (Haigh and Yamanaka \protect\hyperlink{ref-haigh2011}{2011}). Cet algorithme de reconstitution applique un ratio calculé à partir d'une période pour laquelle on dispose de données crédibles sur les débarquements provenant du programme de vérification à quai des pêches à la ligne et à l'hameçon (1997--2005) pour générer une série chronologique des prises par espèce, année, secteur de pêche et zone de gestion (Haigh and Yamanaka \protect\hyperlink{ref-haigh2011}{2011}). Les données « crédibles » sur les débarquements sont tirées des années de référence où la connaissance des prises était considérée comme étant de grande qualité et stable, depuis 1997, avec le début de la présence d'observateurs à bord des chalutiers et le système de quotas individuels des bateaux (Haigh and Yamanaka \protect\hyperlink{ref-haigh2011}{2011}).

La deuxième grande source d'incertitude est l'ampleur des prises non déclarées qui étaient remises à l'eau ou rejetées en mer avant la mise en place du niveau de présence des observateurs de 100 \% en 2006. La reconstitution des prises de Haigh and Yamanaka (\protect\hyperlink{ref-haigh2011}{2011}) suppose qu'il n'y avait pas de rejet avant 1986, année où le permis ZN a été institué. On suppose qu'auparavant, tous les sébastes étaient conservés. Les rejets sont présumés être entièrement déclarés dans les bases de données du MPO depuis 2006 et le niveau de présence des observateurs en mer de 100 \%. Les prises de sébaste aux yeux jaunes non conservées (remises à l'eau ou rejetées) ont été estimées pour chaque pêche à l'aide du ratio des rejets de sébaste aux yeux jaunes par les cibles de débarquement propres à la pêche d'après les données de 2000 à 2004 des registres des observateurs des pêches à la ligne et à l'hameçon. Les prises historiques non déclarées ont ensuite été intégrées à la reconstitution des prises, pour donner un total annuel final.

La série chronologique des prises commerciales utilisée dans cette analyse (figure~\ref{fig:commcatch2} et tableau~\ref{tab:commcatch}) diffère de celle précédemment présentée en 2009 pour plusieurs raisons. Le contrôle continu de la qualité et les mises à jour de la base de données sur les prises de poisson de fond ont entraîné des différences mineures dans les données au fil du temps (Maria Surry, MPO, Station biologique du Pacifique, comm. pers., 9 mars 2020). De plus, une version antérieure de l'algorithme de reconstitution des prises a été utilisée pour élaborer la série chronologique pour l'évaluation des stocks précédente, car la version finale n'avait pas encore été publiée. D'autres améliorations apportées à l'algorithme de reconstitution ont provoqué des changements importants des prises historiques estimées certaines années (Norm Olsen, MPO, Station biologique du Pacifique, comm. pers., 9 mars 2020).

L'algorithme de reconstitution aurait pu être appliqué à toutes les années de la série chronologique jusqu'en 2005 (après quoi la vérification complète en mer et à quai est entrée en vigueur). Toutefois, pour cette analyse, nous avons utilisé les données reconstituées sur les prises de 1918 à 1985 et nous sommes passés aux données sur les prises nominales en 1986. Les prises nominales de 1986 à 2005 ont ensuite été doublées, conformément à l'évaluation précédente (Yamanaka et al. \protect\hyperlink{ref-yamanaka2011}{2011}). Nous avons choisi de doubler les prises nominales plutôt que les prises reconstituées parce que, avant l'évaluation précédente, les représentants de l'industrie nous avaient dit qu'ils n'avaient pas confiance dans la reconstitution des prises entre 1986 et 2005 et que l'échelle des prises non déclarées était probablement égale aux prises débarquées (DFO \protect\hyperlink{ref-dfo2012b}{2012}). Ces indications nous ont amenés à doubler les prises pour ces années (Yamanaka et al. \protect\hyperlink{ref-yamanaka2011}{2011}). Cependant, comme les rejets sont estimés dans le cadre de l'algorithme de reconstitution des prises, dans l'analyse actuelle, nous avons doublé les prises nominales pour la période 1986--2005 (plutôt que les prises reconstituées) afin d'éviter de compter les rejets deux fois. Pour vérifier la sensibilité, nous explorons un scénario de modèle opérationnel où les prises commerciales de 1986 à 2005 n'ont pas été doublées (le scénario « Prises faibles » (2), section~\ref{sec:approach3-reference2}).
\begin{figure}[htb]

{\centering \pdftooltip{\includegraphics[width=0.8\textwidth]{knitr-figs-pdf/commcatch2-1}}{Figure \ref{fig:commcatch2}} 

}

\caption{Prises commerciales par secteur pour le sébaste aux yeux jaunes des eaux intérieures. Cette figure comprend les estimations reconstituées (1918--1985) et nominales (1986--2019) des prises en tonnes.}\label{fig:commcatch2}
\end{figure}
\clearpage
\begin{longtable}[t]{>{\raggedleft\arraybackslash}p{1.0cm}>{\raggedleft\arraybackslash}p{2cm}>{\raggedleft\arraybackslash}p{2cm}>{\raggedleft\arraybackslash}p{2cm}>{\raggedleft\arraybackslash}p{2cm}>{\raggedleft\arraybackslash}p{2cm}}
\caption{\label{tab:commcatch}Prises commerciales par secteur pour le sébaste aux yeux jaunes des eaux intérieures. Le tableau présente les estimations des prises reconstituées (1918--1985) et nominales (1986--2019), en tonnes. Bien que les prises nominales soient indiquées, les prises totales pour chaque année entre 1986 et 2005 ont été doublées dans tous les modèles opérationnels, sauf dans le scénario « Prises faibles », dans un souci de cohérence avec l’évaluation précédente du stock en 2012.}\\
\toprule
\textbf{Année} & \textbf{Chalut} & \textbf{Flétan} & \textbf{Aiguillat commun et morue-lingue} & \textbf{Sébaste à la ligne et à l’hameçon} & \textbf{Total}\\
\midrule
\endfirsthead
\caption*{}\\
\toprule
\textbf{Année} & \textbf{Chalut} & \textbf{Flétan} & \textbf{Aiguillat commun et morue-lingue} & \textbf{Sébaste à la ligne et à l’hameçon} & \textbf{Total}\\
\midrule
\endhead
\
\endfoot
\bottomrule
\endlastfoot
1918 & 0.00 & 3.40 & 4.90 & 8.80 & 17.10\\
1919 & 0.00 & 8.50 & 12.00 & 22.00 & 42.50\\
1920 & 0.00 & 4.30 & 6.00 & 11.00 & 21.30\\
1921 & 0.00 & 3.70 & 5.20 & 9.50 & 18.40\\
1922 & 0.00 & 4.60 & 6.50 & 12.00 & 23.10\\
1923 & 0.00 & 4.50 & 6.30 & 11.00 & 21.80\\
1924 & 0.00 & 5.10 & 7.20 & 13.00 & 25.30\\
1925 & 0.00 & 4.40 & 6.20 & 11.00 & 21.60\\
1926 & 0.00 & 5.00 & 7.10 & 13.00 & 25.10\\
1927 & 0.00 & 5.00 & 7.10 & 13.00 & 25.10\\
1928 & 0.00 & 5.10 & 7.30 & 13.00 & 25.40\\
1929 & 0.00 & 6.70 & 9.50 & 17.00 & 33.20\\
1930 & 0.00 & 6.00 & 8.60 & 16.00 & 30.60\\
1931 & 0.00 & 4.00 & 5.60 & 10.00 & 19.60\\
1932 & 0.00 & 4.50 & 6.40 & 12.00 & 22.90\\
1933 & 0.00 & 2.20 & 3.10 & 5.70 & 11.00\\
1934 & 0.00 & 2.60 & 3.70 & 6.70 & 13.00\\
1935 & 0.00 & 3.30 & 4.70 & 8.60 & 16.60\\
1936 & 0.00 & 3.60 & 5.10 & 9.30 & 18.00\\
1937 & 0.00 & 2.80 & 4.00 & 7.30 & 14.10\\
1938 & 0.00 & 10.00 & 14.00 & 25.00 & 49.00\\
1939 & 0.00 & 1.90 & 2.70 & 4.80 & 9.40\\
1940 & 0.00 & 2.00 & 2.90 & 5.30 & 10.20\\
1941 & 0.00 & 1.30 & 1.80 & 3.20 & 6.30\\
1942 & 0.00 & 2.90 & 4.10 & 7.50 & 14.50\\
1943 & 0.00 & 17.00 & 24.00 & 43.00 & 84.00\\
1944 & 0.00 & 25.00 & 36.00 & 64.00 & 125.00\\
1945 & 0.00 & 27.00 & 38.00 & 69.00 & 134.00\\
1946 & 0.00 & 18.00 & 26.00 & 46.00 & 90.00\\
1947 & 0.00 & 5.80 & 8.10 & 15.00 & 28.90\\
1948 & 0.00 & 8.80 & 12.00 & 23.00 & 43.80\\
1949 & 0.00 & 12.00 & 17.00 & 30.00 & 59.00\\
1950 & 0.00 & 5.00 & 7.00 & 13.00 & 25.00\\
1951 & 0.00 & 3.60 & 5.10 & 9.30 & 18.00\\
1952 & 0.00 & 2.80 & 3.90 & 7.10 & 13.80\\
1953 & 0.00 & 5.80 & 8.30 & 15.00 & 29.10\\
1954 & 0.00 & 3.60 & 5.10 & 9.30 & 18.00\\
1955 & 0.00 & 3.60 & 5.10 & 9.20 & 17.90\\
1956 & 0.00 & 3.40 & 4.80 & 8.80 & 17.00\\
1957 & 0.00 & 5.90 & 8.40 & 15.00 & 29.30\\
1958 & 0.00 & 8.60 & 12.00 & 22.00 & 42.60\\
1959 & 0.00 & 8.80 & 13.00 & 23.00 & 44.80\\
1960 & 0.00 & 7.20 & 10.00 & 18.00 & 35.20\\
1961 & 0.00 & 5.30 & 7.60 & 14.00 & 26.90\\
1962 & 0.00 & 8.60 & 12.00 & 22.00 & 42.60\\
1963 & 0.00 & 6.60 & 9.30 & 17.00 & 32.90\\
1964 & 0.00 & 4.00 & 5.60 & 10.00 & 19.60\\
1965 & 0.00 & 3.60 & 5.10 & 9.20 & 17.90\\
1966 & 0.00 & 2.90 & 4.10 & 7.40 & 14.40\\
1967 & 0.00 & 4.50 & 6.30 & 11.00 & 21.80\\
1968 & 0.00 & 4.80 & 6.80 & 12.00 & 23.60\\
1969 & 0.00 & 5.60 & 7.90 & 14.00 & 27.50\\
1970 & 0.00 & 6.80 & 10.00 & 18.00 & 34.80\\
1971 & 0.00 & 5.80 & 8.30 & 15.00 & 29.10\\
1972 & 0.00 & 6.50 & 9.10 & 17.00 & 32.60\\
1973 & 0.00 & 7.90 & 11.00 & 20.00 & 38.90\\
1974 & 0.00 & 3.90 & 5.50 & 10.00 & 19.40\\
1975 & 0.00 & 3.10 & 4.40 & 8.00 & 15.50\\
1976 & 0.00 & 3.80 & 5.40 & 10.00 & 19.20\\
1977 & 0.10 & 11.00 & 15.00 & 27.00 & 53.10\\
1978 & 0.20 & 12.00 & 17.00 & 31.00 & 60.20\\
1979 & 0.00 & 19.00 & 27.00 & 49.00 & 95.00\\
1980 & 0.00 & 14.00 & 20.00 & 36.00 & 70.00\\
1981 & 0.00 & 16.00 & 23.00 & 42.00 & 81.00\\
1982 & 5.90 & 22.00 & 14.00 & 13.00 & 54.90\\
1983 & 7.90 & 23.00 & 14.00 & 6.60 & 51.50\\
1984 & 30.00 & 27.00 & 8.40 & 9.40 & 74.80\\
1985 & 68.00 & 34.00 & 7.60 & 10.00 & 119.60\\
1969 & 0.04 & 0.00 & 0.00 & 0.00 & 0.04\\
1977 & 0.10 & 0.00 & 0.00 & 0.00 & 0.10\\
1978 & 0.18 & 0.00 & 0.00 & 0.00 & 0.18\\
1979 & 0.02 & 0.00 & 0.00 & 0.00 & 0.02\\
1980 & 0.00 & 0.00 & 0.00 & 0.00 & 0.00\\
1983 & 0.03 & 0.00 & 0.00 & 0.00 & 0.03\\
1984 & 0.00 & 0.00 & 0.00 & 0.00 & 0.00\\
1986 & 0.00 & 0.00 & 0.00 & 157.89 & 157.89\\
1987 & 0.35 & 0.00 & 0.00 & 97.73 & 98.08\\
1988 & 0.01 & 0.00 & 0.00 & 128.31 & 128.32\\
1989 & 0.01 & 0.00 & 0.00 & 119.42 & 119.43\\
1990 & 0.00 & 0.00 & 0.00 & 128.53 & 128.53\\
1991 & 0.00 & 0.00 & 0.00 & 60.63 & 60.63\\
1992 & 0.00 & 0.00 & 0.00 & 25.55 & 25.55\\
1993 & 0.01 & 0.00 & 0.00 & 41.59 & 41.60\\
1994 & 0.36 & 0.00 & 0.00 & 88.81 & 89.17\\
1995 & 0.05 & 0.65 & 0.00 & 34.24 & 34.94\\
1996 & 0.04 & 1.27 & 0.06 & 25.22 & 26.59\\
1997 & 0.00 & 2.44 & 0.05 & 23.09 & 25.58\\
1998 & 0.01 & 6.34 & 0.23 & 29.04 & 35.62\\
1999 & 0.00 & 1.57 & 0.07 & 23.04 & 24.68\\
2000 & 0.00 & 0.44 & 0.00 & 26.79 & 27.23\\
2001 & 0.01 & 0.83 & 0.21 & 23.23 & 24.28\\
2002 & 0.00 & 0.02 & 0.03 & 3.32 & 3.37\\
2003 & 0.01 & 0.00 & 1.48 & 3.51 & 5.00\\
2004 & 0.00 & 0.19 & 1.77 & 2.08 & 4.04\\
2005 & 0.00 & 0.02 & 3.45 & 2.23 & 5.70\\
2006 & 0.00 & 0.49 & 2.22 & 1.06 & 3.77\\
2007 & 0.01 & 1.74 & 2.43 & 3.34 & 7.52\\
2008 & 0.00 & 2.26 & 2.80 & 4.40 & 9.46\\
2009 & 0.00 & 0.93 & 2.85 & 2.67 & 6.45\\
2010 & 0.00 & 1.12 & 2.56 & 2.84 & 6.52\\
2011 & 0.00 & 1.26 & 1.48 & 4.00 & 6.74\\
2012 & 0.00 & 1.23 & 1.26 & 2.61 & 5.10\\
2013 & 0.00 & 0.28 & 1.34 & 3.51 & 5.13\\
2014 & 0.00 & 1.03 & 0.55 & 3.48 & 5.06\\
2015 & 0.00 & 0.19 & 1.80 & 3.51 & 5.50\\
2016 & 0.04 & 0.49 & 0.62 & 3.18 & 4.33\\
2017 & 0.01 & 1.26 & 0.00 & 5.75 & 7.02\\
2018 & 0.02 & 0.29 & 0.00 & 2.42 & 2.73\\
2019 & 0.00 & 0.07 & 0.01 & 2.71 & 2.79\\
2020 & 0.01 & 0.00 & 0.00 & 0.32 & 0.33\\*
\end{longtable}
\clearpage

\hypertarget{sec:rec-catch-data}{%
\subsection{DONNÉES SUR LES PRISES RÉCRÉATIVES}\label{sec:rec-catch-data}}

En 2012, le MPO a établi un relevé sur Internet destiné aux détenteurs de permis de pêche en eaux de marées à l'échelle de la côte (iRec) qui permet de recueillir des données sur le sébaste aux yeux jaunes (DFO \protect\hyperlink{ref-dfo2015}{2015}). Toutefois, on n'a pas étalonné les résultats de ce relevé pour tenir compte des incertitudes comme le biais de non-réponse. C'est pourquoi les données iRec n'ont pas été incluses dans cette analyse.

\hypertarget{sec:recon-rec-catch-data}{%
\subsubsection{Prises récréatives historiques reconstituées}\label{sec:recon-rec-catch-data}}

Les prises récréatives historiques, avant 1982, ont été reconstituées pour l'évaluation précédente d'après les tendances de l'effort de pêche dégagées d'entrevues avec les propriétaires d'un camp de pêche récréative (Yamanaka et al. \protect\hyperlink{ref-yamanaka2011}{2011}). Nous avons utilisé la même série chronologique des prises récréatives reconstituées pour l'analyse actuelle (tableau~\ref{tab:rectable}).

\hypertarget{sec:creel-catch-data}{%
\subsubsection{Données des relevés par interrogation de pêcheurs, de 1982 à 2019}\label{sec:creel-catch-data}}

Les prises annuelles de sébaste aux yeux jaunes des eaux intérieures dans la pêche récréative sont estimées par les relevés par interrogation de pêcheurs dans le détroit de Georgie (DG) et du nord de l'île de Vancouver (NIV) dans tous les SGPP (figure~\ref{fig:map-4B}). Les relevés portent sur les SGPP 12--20, 28 et 29 (Zetterberg and Carter \protect\hyperlink{ref-zetterberg2010}{2010}). Les prises de sébaste ont été enregistrées dans les secteurs 13--19, 28 et 29 depuis 1982, mais n'étaient pas dénombrées par espèce avant 2000. Dans le SGPP 12, les sébastes sont dénombrés par espèce depuis 2000, sans enregistrement avant 2000 (Zetterberg and Carter \protect\hyperlink{ref-zetterberg2010}{2010}).

Nous avons suivi la même méthode que celle de Yamanaka et al. (\protect\hyperlink{ref-yamanaka2011}{2011}) pour estimer les prises récréatives de sébaste aux yeux jaunes des eaux intérieures de 1982 à 1999. Tout d'abord, pour tous les SGPP autres que le SGPP 12, nous avons calculé la proportion moyenne des prises de sébaste aux yeux jaunes par rapport aux prises totales de sébaste pour chaque SGPP en 2000 et en 2001. Nous avons ensuite utilisé les proportions moyennes pour dériver les estimations des prises de sébaste aux yeux jaunes des prises totales de sébastes par SGPP de 1982 à 1999. L'évaluation précédente supposait que la proportion des prises de sébaste aux yeux jaunes dans le SGPP 12, sur le total des prises de sébaste aux yeux jaunes dans le détroit de Georgie, demeurerait relativement constante dans le temps. Par conséquent, pour estimer les prises de sébaste aux yeux jaunes dans le SGPP 12 pour les années 1982--1999, nous avons calculé la proportion de sébastes aux yeux jaunes capturés dans le SGPP 12, sur le total de sébastes aux yeux jaunes capturés dans le détroit de Georgie en 2000 et en 2001. Nous avons ensuite multiplié la proportion moyenne pour 2000 et 2001 par le total des prises de sébaste aux yeux jaunes estimées pour le reste du détroit de Georgie (somme des secteurs 13--19, 28 et 29) afin d'estimer les prises de sébaste aux yeux jaunes dans le SGPP 12 par année (tableau~\ref{tab:recbyarea}). Pour assurer la conformité à l'évaluation précédente, nous avons appliqué un ajustement de 1,09 à l'effort annuel total pour tenir compte du manque d'enregistrements dans le SGPP 12, où l'effort n'a pas été enregistré avant 2000. Nous avons converti les nombres de sébastes en poids en multipliant par 2,49 kg le poids moyen de l'échantillon de sébaste aux yeux jaunes prélevé dans les relevés par interrogation de pêcheurs entre 2000 et 2008.

Nous n'avons pas élaboré d'indice des CPUE pour la pêche récréative, bien que des données récentes sur les relevés par interrogation de pêcheurs soient accessibles. Depuis l'imposition de procédures de gestion visant la conservation des sébastes (tableau~\ref{tab:rec-mgt-changes}), une tendance à l'évitement actif du sébaste a été observée dans les pêches récréatives. De ce fait, nous craignons qu'une série de CPUE pour la pêche récréative ne tienne pas compte des changements de l'abondance et ne soit donc trompeuse aux fins de l'évaluation.

\clearpage
\begin{figure}[htb]

{\centering \pdftooltip{\includegraphics[width=0.8\textwidth]{knitr-figs-pdf/reccatch-1}}{Figure \ref{fig:reccatch}} 

}

\caption{Prises récréatives de sébaste aux yeux jaunes des eaux intérieures. La ligne noire indique les prises reconstituées et les barres sont les données des relevés par interrogation de pêcheurs. Les données sont une combinaison des prises reconstituées (1918--1981), des prises analysées à partir du total des prises de sébaste dans les relevés par interrogation de pêcheurs (1982--1999) et des prises provenant de relevés sur certaines espèces par interrogation de pêcheurs (1982--2019).}\label{fig:reccatch}
\end{figure}
\begin{longtable}[t]{ccc}
\caption{\label{tab:rectable}Prises récréatives de sébaste aux yeux jaunes des eaux intérieures. Les données sont une combinaison des prises reconstituées (1918--1981), des prises analysées à partir du total des prises de sébaste dans les relevés par interrogation de pêcheurs (1982--1999) et des prises provenant de relevés sur certaines espèces par interrogation de pêcheurs (1982--2019).}\\
\toprule
\textbf{Année} & \textbf{Prises (t)} & \textbf{Effort (sorties de bateaux)}\\
\midrule
\endfirsthead
\caption*{}\\
\toprule
\textbf{Année} & \textbf{Prises (t)} & \textbf{Effort (sorties de bateaux)}\\
\midrule
\endhead
\
\endfoot
\bottomrule
\endlastfoot
1918 & 0.9 & 16600\\
1919 & 0.9 & 16600\\
1920 & 0.9 & 16600\\
1921 & 0.9 & 16600\\
1922 & 0.9 & 16600\\
1923 & 0.9 & 16600\\
1924 & 0.9 & 16600\\
1925 & 0.9 & 16600\\
1926 & 0.9 & 16600\\
1927 & 0.9 & 16600\\
1928 & 0.9 & 16600\\
1929 & 0.9 & 16600\\
1930 & 0.9 & 16600\\
1931 & 0.9 & 16600\\
1932 & 0.9 & 16600\\
1933 & 0.9 & 16600\\
1934 & 0.9 & 16600\\
1935 & 0.9 & 16600\\
1936 & 0.9 & 16600\\
1937 & 0.9 & 16600\\
1938 & 0.9 & 16600\\
1939 & 0.9 & 16600\\
1940 & 0.9 & 16600\\
1941 & 0.9 & 16600\\
1942 & 0.9 & 16600\\
1943 & 0.9 & 16600\\
1944 & 0.9 & 16600\\
1945 & 0.9 & 16600\\
1946 & 1 & 19800\\
1947 & 2 & 39500\\
1948 & 3.1 & 59300\\
1949 & 4.1 & 79000\\
1950 & 5.1 & 98800\\
1951 & 6.1 & 118600\\
1952 & 7.2 & 138300\\
1953 & 8.2 & 158100\\
1954 & 9.2 & 177800\\
1955 & 10.2 & 197600\\
1956 & 11.3 & 217400\\
1957 & 12.3 & 237100\\
1958 & 13.3 & 256900\\
1959 & 14.3 & 276600\\
1960 & 15.4 & 296400\\
1961 & 17.2 & 332800\\
1962 & 17.2 & 332800\\
1963 & 17.2 & 332800\\
1964 & 17.2 & 332800\\
1965 & 17.2 & 332800\\
1966 & 17.2 & 332800\\
1967 & 17.2 & 332800\\
1968 & 17.2 & 332800\\
1969 & 17.2 & 332800\\
1970 & 18.2 & 350300\\
1971 & 19.1 & 367800\\
1972 & 20 & 385300\\
1973 & 20.9 & 402900\\
1974 & 21.8 & 420400\\
1975 & 22.7 & 437900\\
1976 & 23.6 & 455400\\
1977 & 24.5 & 472900\\
1978 & 25.4 & 490400\\
1979 & 26.3 & 507900\\
1980 & 27.2 & 525500\\
1981 & 15.9 & 306500\\
1982 & 33.4 & 609738\\
1983 & 29.5 & 581764\\
1984 & 22 & 709686\\
1985 & 24.8 & 685082\\
1986 & 32.1 & 635438\\
1987 & 21.6 & 642698\\
1988 & 32.9 & 714448\\
1989 & 36.2 & 657638\\
1990 & 29.3 & 572527\\
1991 & 33.3 & 493093\\
1992 & 26.3 & 500668\\
1993 & 17 & 542848\\
1994 & 26.6 & 480411\\
1995 & 21.2 & 352770\\
1996 & 19.8 & 314722\\
1997 & 15.6 & 297399\\
1998 & 14.3 & 181874\\
1999 & 11.4 & 178133\\
2000 & 16 & 201870\\
2001 & 22 & 206345\\
2002 & 8.4 & 217408\\
2003 & 9.5 & 197227\\
2004 & 7.7 & 148898\\
2005 & 3.9 & 125019\\
2006 & 6.2 & 119256\\
2007 & 2.4 & 128332\\
2008 & 4.4 & 110568\\
2009 & 4.1 & 119343\\
2010 & 4.6 & 111307\\
2011 & 4.8 & 113544\\
2012 & 5.1 & --\\
2013 & 11.1 & 164466\\
2014 & 1.9 & 132320\\
2015 & 6.1 & 159424\\
2016 & 8.9 & 143219\\
2017 & 10.3 & 168385\\
2018 & 7.8 & 199023\\
2019 & 4.9 & 111171\\*
\end{longtable}
\clearpage
\begin{longtable}[t]{>{\raggedright\arraybackslash}p{0.75cm}>{\raggedright\arraybackslash}p{0.75cm}>{\raggedright\arraybackslash}p{0.75cm}>{\raggedright\arraybackslash}p{0.75cm}>{\raggedright\arraybackslash}p{0.75cm}>{\raggedright\arraybackslash}p{0.75cm}>{\raggedright\arraybackslash}p{0.75cm}>{\raggedright\arraybackslash}p{0.75cm}>{\raggedright\arraybackslash}p{0.75cm}>{\raggedright\arraybackslash}p{0.75cm}>{\raggedright\arraybackslash}p{0.75cm}>{\raggedright\arraybackslash}p{0.75cm}>{\raggedright\arraybackslash}p{0.75cm}}
\caption{\label{tab:recbyarea}Estimations des prises récréatives de sébaste aux yeux jaunes (en tonnes) tirées des relevés par interrogation de pêcheurs dans les eaux intérieures du détroit de Georgie par zone statistique (SGPP) et effort total dans 10 000 sorties en bateau par année de 1982 à 2019. Le nombre de poissons a été converti en poids à l’aide d’un facteur de 2,49 kg (poids moyen du sébaste aux yeux jaunes dans les relevés par interrogation de pêcheurs 2000--2008).}\\
\toprule
\textbf{Annee} & \textbf{PFMA 13} & \textbf{PFMA 14} & \textbf{PFMA 15} & \textbf{PFMA 16} & \textbf{PFMA 17} & \textbf{PFMA 18} & \textbf{PFMA 19} & \textbf{PFMA 20} & \textbf{PFMA 28} & \textbf{PFMA 29} & \textbf{PFMA 12} & \textbf{Effort}\\
\midrule
\endfirsthead
\caption*{}\\
\toprule
\textbf{Annee} & \textbf{PFMA 13} & \textbf{PFMA 14} & \textbf{PFMA 15} & \textbf{PFMA 16} & \textbf{PFMA 17} & \textbf{PFMA 18} & \textbf{PFMA 19} & \textbf{PFMA 20} & \textbf{PFMA 28} & \textbf{PFMA 29} & \textbf{PFMA 12} & \textbf{Effort}\\
\midrule
\endhead
\
\endfoot
\bottomrule
\endlastfoot
1982 & 1.7 & 3.5 & 2.4 & 20.4 & 2.1 & 0.4 & 1.2 & 0.3 & 0.2 & 1 & -- & 61\\
1983 & 3.2 & 3.5 & 2.5 & 13.9 & 2.3 & 0.3 & 0.9 & 1.5 & 0.3 & 1.1 & -- & 58\\
1984 & 2 & 3 & 2.9 & 6.1 & 4.4 & 0.3 & 0.8 & 1.1 & 0.3 & 1.2 & -- & 71\\
1985 & 1.3 & 2.6 & 1.1 & 14.6 & 2.6 & 0.2 & 0.8 & 0.6 & 0.2 & 1 & -- & 69\\
1986 & 1.9 & 4.3 & 1.8 & 18.5 & 2.3 & 0.2 & 1 & 0.9 & 0.2 & 1.1 & -- & 64\\
1987 & 1.4 & 4.7 & 2 & 7.8 & 2.9 & 0.2 & 0.8 & 1.1 & 0.1 & 0.5 & -- & 64\\
1988 & 2.2 & 6.1 & 1.9 & 14.4 & 3.8 & 0.3 & 1 & 1.5 & 0.1 & 1.6 & -- & 71\\
1989 & 1.6 & 6.6 & 2.1 & 18.3 & 4.1 & 0.3 & 1.6 & -- & 0.1 & 1.4 & -- & 66\\
1990 & 1.6 & 4.7 & 1.7 & 16.1 & 2 & 0.1 & 0.9 & 0.9 & 0.1 & 1.2 & -- & 57\\
1991 & 1.5 & 4.8 & 1.5 & 18 & 2.5 & 0.1 & 0.7 & 0.4 & 0.3 & 3.5 & -- & 49\\
1992 & 1.3 & 2.9 & 0.8 & 16.5 & 2.2 & 0.2 & 0.7 & 0.5 & 0.2 & 1 & -- & 50\\
1993 & 1.4 & 1.9 & 0.8 & 7.9 & 1.9 & 0.1 & 0.8 & 0.6 & 0.1 & 1.5 & -- & 54\\
1994 & 2.5 & 5.1 & 1.9 & 11.1 & 2.5 & 0.1 & 0.8 & 0.5 & 0.3 & 1.9 & -- & 48\\
1995 & 1.7 & 2.6 & 1.5 & 11.1 & 1.9 & 0.1 & 0.5 & 0.4 & 0.1 & 1.3 & -- & 35\\
1996 & 2 & 1.1 & 1.3 & 12.5 & 0.8 & 0.1 & 0.7 & 0.4 & 0.2 & 0.8 & -- & 31\\
1997 & 1.7 & 1.3 & 1.5 & 8.1 & 1.3 & 0.1 & 0.4 & 0.3 & 0.2 & 0.8 & -- & 30\\
1998 & 2.6 & 0.8 & 1.5 & 6.7 & 1.1 & 0.1 & 0.5 & 0.8 & 0.1 & 0.2 & -- & 18\\
1999 & 1.9 & 0.5 & 0.4 & 6.7 & 0.7 & 0 & 0.3 & 0.6 & 0.1 & 0.1 & -- & 18\\
2000 & 1 & 1 & 0.6 & 5.4 & 2.2 & 0 & 0.8 & 1.1 & 0.2 & 0.2 & 3.7 & 20\\
2001 & 1.9 & 3.3 & 1.4 & 9.6 & 2.4 & 0.1 & 0.1 & 0.7 & -- & 0.5 & 1.9 & 21\\
2002 & 0.5 & 4.3 & 0.9 & 1.6 & 0.5 & -- & 0.1 & 0.1 & 0.1 & 0.1 & 0.2 & 22\\
2003 & 0 & 0.3 & 0.3 & 6.5 & 0.6 & 0 & 0 & 0.1 & 0.1 & 0.2 & 1.2 & 20\\
2004 & 0.2 & 4.2 & 0.4 & 1.8 & 0.2 & -- & 0 & 0.2 & 0 & 0.1 & 0.4 & 15\\
2005 & 0.3 & 0.6 & 0 & 0.2 & 0.9 & -- & 0 & 0.6 & 0 & 0 & 1.2 & 13\\
2006 & 0.2 & 1.7 & 0.2 & 0.4 & 1.1 & 0.1 & 0.2 & 0.5 & -- & -- & 1.7 & 12\\
2007 & 0.2 & 0 & 0.1 & 0.3 & 0.2 & -- & 0.1 & 0.5 & 0 & 0 & 0.9 & 13\\
2008 & 0.1 & 0.2 & 0.4 & 1 & 0.2 & -- & 0 & 0.8 & 0 & 0 & 1.7 & 11\\
2009 & 0.2 & 0.1 & 0.4 & 0.5 & 1.1 & -- & -- & 0.6 & 0.2 & 0 & 1.2 & 12\\
2010 & 0.1 & 0.7 & 0.5 & 1 & 0.3 & 0 & 0 & 0.3 & -- & 0 & 1.7 & 11\\
2011 & 0.4 & 1 & 0.2 & 0.5 & 0.3 & -- & -- & 0.4 & -- & 0.4 & 1.6 & 11\\
2012 & 0.3 & 0.5 & 0.2 & 0.2 & 0.5 & 0 & 0 & 0.6 & -- & 0.1 & 2.7 & 13\\
2013 & 0.6 & 1.5 & 0.8 & 2.4 & 4.4 & 0 & 0.1 & 0.3 & 0.1 & 0.2 & 0.7 & 16\\
2014 & 0.2 & 0.2 & -- & -- & 0.3 & 0.1 & 0.1 & 0.3 & -- & -- & 0.7 & 13\\
2015 & 0.7 & 0.5 & 1 & 1.5 & 0.9 & -- & 0 & 0.3 & -- & -- & 1.2 & 16\\
2016 & 0.2 & 1.8 & 1.9 & 1.7 & 2 & 0 & 0 & 0.6 & -- & -- & 0.7 & 14\\
2017 & 0.2 & 3.1 & 1.8 & 2.1 & 1.3 & 0.1 & -- & 0.6 & -- & 0 & 1.1 & 17\\
2018 & 0.4 & 0.7 & 2.1 & 1.6 & 1.4 & 0.1 & 0.1 & 0.1 & 0 & 0.1 & 1.1 & 20\\
2019 & 0.2 & 0.6 & 1.5 & 1.1 & 0.4 & -- & 0.1 & 0.7 & -- & -- & 0.3 & 11\\*
\end{longtable}
\clearpage

\hypertarget{sec:fsc-catch-data}{%
\subsection{PRISES À DES FINS ALIMENTAIRES, SOCIALES ET RITUELLES (ASR)}\label{sec:fsc-catch-data}}

Le sébaste aux yeux jaunes est une importante source de nourriture traditionnelle pour les Premières Nations de la côte de la Colombie-Britannique (Eckert et al. \protect\hyperlink{ref-eckert2018}{2018}), y compris dans les eaux intérieures de la zone 4B. Les prises à des fins ASR totales de sébaste aux yeux jaunes ne sont accessibles ni pour la période historique, ni pour la période contemporaine, et les données accessibles ne sont pas résolues au niveau de l'espèce (M. Fetterly, MPO, Analyse des politiques et soutien aux traités, comm. pers., 7 novembre 2019 et A. Rushton, MPO, Gestion des pêches de la côte sud, comm. pers., 7 février 2020). Pour tenir compte des prises à des fins ASR dans la dernière évaluation des stocks, Yamanaka et al. (\protect\hyperlink{ref-yamanaka2011}{2011}) ont utilisé un taux de consommation (0,23 kg/année/personne), qui représentait la moitié du taux de consommation déterminé dans une étude sur l'alimentation traditionnelle dans le sud-est de l'Alaska. Ils ont appliqué le taux de consommation aux populations des Premières Nations proches de la zone 4B pour estimer la consommation totale au cours de la série chronologique (de 1918 à 2009). Cette approche suppose que le taux de consommation de sébaste aux yeux jaunes par les Premières Nations est demeuré constant, mais on sait que la colonisation européenne a eu une incidence sur la plupart des aspects de la société autochtone durant cette période. Une baisse de la quantité de poisson et de fruits de mer consommée par les Premières Nations en Colombie-Britannique a été attribuée à de nombreux facteurs sociaux, écologiques et économiques, notamment la perte de territoires traditionnels, la diminution de la transmission du savoir traditionnel, et des obstacles comme la pauvreté qui rend l'achat de bateaux et d'engins de pêche inaccessible pour de nombreuses communautés (Marushka et al. \protect\hyperlink{ref-marushka2019}{2019}). En ce qui concerne la partie sud de notre zone d'étude, les peuples Salish du littoral ont vu leur relation avec les ressources marines s'éroder en raison du développement des pêches commerciales et récréatives, ainsi que des politiques et des décisions politiques (Ayers et al. \protect\hyperlink{ref-ayers2012}{2012}). Nous n'avons donc pas suivi les méthodes utilisées dans Yamanaka et al. (\protect\hyperlink{ref-yamanaka2011}{2011}).

Les seules données sur les pêches à des fins ASR accessibles proviennent du Programme de vérification à quai entre 2006 et 2019 (tableau~\ref{tab:fsc-catch}). Ces données ont été recueillies dans le cadre de sorties de « pêche double », qui ont lieu lorsque les pêcheurs autochtones choisissent de conserver à des fins ASR une partie des prises obtenues pendant une sortie de pêche commerciale. Les prises commerciales et à des fins ASR sont surveillées pendant le déchargement. Entre 0 et 0,8 tonne, soit une moyenne de 5,6 \% du total des prises commerciales, a été débarquée dans le cadre de sorties de pêche double durant cette période. Les prises à des fins ASR de ces sorties de pêche double sont incluses dans les totaux annuels des prises commerciales dans les bases de données du secteur du poisson de fond. Les données sur les prises du Programme de vérification à quai ne peuvent être résolues qu'au niveau de la sortie plutôt qu'à celui de la calée, de sorte que certaines des données sur la pêche double peuvent provenir de l'extérieur de la zone 4B (c.-à-d.~inclure des prises de sébaste aux yeux jaunes des eaux extérieures). Pour régler ce problème, si plus de 50 \% des calées d'une sortie ont eu lieu dans la zone 4B, nous les avons incluses dans les données sur les prises commerciales pour la zone 4B. À l'inverse, nous avons exclu les sorties dont au moins 50 \% des calées avaient été effectuées à l'extérieur de la zone 4B. La plupart des sorties de pêche double ont eu lieu dans la partie nord de la zone d'étude parce que c'est aussi là que se pratique actuellement la plus grande partie de la pêche commerciale du sébaste aux yeux jaunes dans la zone 4B. Les pêcheurs autochtones des Premières Nations membres représentés par la A-Tlegay Fisheries Society capturent essentiellement des sébastes aux yeux jaunes lors de sorties de pêche double (C. Rusel, comm. pers., 8 novembre 2019).

Dans la partie sud de la zone d'étude, les pêcheurs autochtones ont une faible capacité commerciale. Les prises à des fins ASR dans le détroit de Georgie proviennent donc principalement de petits bateaux de pêche récréative (C. Ayers, comm. pers., 7 novembre 2019; B. Bocking, comm. pers., 7 novembre 2019). Une partie de l'effort à des fins ASR des petits bateaux sera enregistrée dans les données sur les activités récréatives du programme de relevé par interrogation de pêcheurs. Bien que les pêcheurs à des fins ASR ne soient pas liés par les limites de prises ou les fermetures des pêches récréatives, leurs bateaux seront comptés dans la partie aérienne du relevé par interrogation de pêcheurs et contribueront donc aux estimations élargies des prises récréatives. Toutefois, la proportion de pêcheurs à des fins ASR rencontrés par le vérificateur à quai n'était pas facilement accessible dans la base de données des pêcheurs récréatifs (CREST) {[}P. Zetterberg, MPO, comm. pers., 29 novembre 2019{]}.

Comme nous l'avons montré, il y a peu d'information accessible pour aider à quantifier les prises à des fins ASR de sébaste aux yeux jaunes des eaux intérieures. Sans des informations plus détaillées, il n'est pas possible d'estimer de façon fiable l'impact des prises à des fins ASR sur les résultats de cette analyse. Une plus grande collaboration avec les Premières Nations pourrait aider à régler certains de ces problèmes de données et devrait être une priorité pour les analyses futures.

\clearpage
\begin{longtable}[t]{lrrrr}
\caption{\label{tab:fsc-catch}Prises à des fins ASR de sébaste aux yeux jaunes des eaux intérieures en proportion du total des prises commerciales déclarées aux observateurs à quai lors de sorties de pêche double.}\\
\toprule
\textbf{Année} & \textbf{ASR} & \textbf{Commerciales} & \textbf{Total} & \textbf{Pourcentage ASR}\\
\midrule
\endfirsthead
\caption*{}\\
\toprule
\textbf{Année} & \textbf{ASR} & \textbf{Commerciales} & \textbf{Total} & \textbf{Pourcentage ASR}\\
\midrule
\endhead
\
\endfoot
\bottomrule
\endlastfoot
2006 & 0.0 & 2.3 & 3.1 & 0.0\\
2007 & 0.7 & 4.3 & 5.0 & 13.8\\
2008 & 0.4 & 6.6 & 6.9 & 5.1\\
2009 & 0.3 & 4.6 & 4.9 & 5.2\\
2010 & 0.3 & 5.4 & 5.7 & 6.0\\
2011 & 0.3 & 4.0 & 4.3 & 6.5\\
2012 & 0.6 & 3.3 & 3.8 & 14.9\\
2013 & 0.0 & 5.0 & 5.0 & 0.1\\
2014 & 0.0 & 4.2 & 4.3 & 0.8\\
2015 & 0.0 & 4.2 & 4.2 & 0.0\\
2016 & 0.3 & 4.2 & 4.5 & 5.7\\
2017 & 0.8 & 5.9 & 6.7 & 12.0\\
2018 & 0.2 & 2.1 & 2.3 & 8.8\\
2019 & 0.0 & 2.3 & 2.3 & 0.0\\
Total & 3.8 & 58.3 & 62.1 & 6.1\\
Mean & 0.3 & 4.2 & 4.5 & 5.6\\*
\end{longtable}
\clearpage

\hypertarget{sec:management-changes}{%
\subsection{CHRONOLOGIE DES CHANGEMENTS DE GESTION}\label{sec:management-changes}}
\begin{longtable}[t]{>{\raggedright\arraybackslash}p{2.5cm}>{\raggedright\arraybackslash}p{1.75cm}>{\raggedright\arraybackslash}p{7.5cm}}
\caption{\label{tab:comm-mgt-changes}Historique des changements apportés à la gestion de la pêche commerciale du sébaste dans la zone 4B de 1986 à 2019.}\\
\toprule
\textbf{Année} & \textbf{Zone} & \textbf{Mesure de gestion}\\
\midrule
\endfirsthead
\caption*{}\\
\toprule
\textbf{Année} & \textbf{Zone} & \textbf{Mesure de gestion}\\
\midrule
\endhead
\
\endfoot
\bottomrule
\endlastfoot
1986 & Coastwide & Introduced a category ZN license for the directed hook-and-line rockfish fishery with a voluntary logbook program\\
1986 & Inside & Feb 15 to Apr 15 closure\\
1987 & Inside & Jan 1 to Apr 15 closure\\
1987 & Inside & Provisional 75-metric-ton quota, area 12\\
1988 & Inside & Year-round commercial closure, area 13 Discovery Pass\\
1988 & Inside & Jan 1 to Apr 30 closure\\
1990 & Inside & Jan 1 to Apr 30 and Nov 1 to Dec 31 closure\\
1991 & Coastwide & Area licensing,  592 inside\\
1991 & Inside & Trawl closure\\
1991 & Inside & Live rockfish fishery only\\
1991 & Inside & Jan 1 to May 14 closure, with no incidental rockfish catch allowances\\
1991 & Inside & 2-3-d opening in area 13 Discovery Pass\\
1991 & Coastwide & Limited-entry licensing program was announced\\
1992 & Inside & Limited-entry licensing with 74 eligible inside licenses\\
1993 & Coastwide & TAC quota management for red snapper and other rockfish by five management regions\\
1993 & Coastwide & Region and time closures\\
1994 & Coastwide & User-pay logbook program\\
1994 & Coastwide & Trip limits for trawl species\\
1994 & Coastwide & Incidental catch allowances\\
1995 & Coastwide & User-pay dockside monitoring program\\
1995 & Coastwide & Aggregate species quota management for yelloweye rockfish\\
1995 & Coastwide & Monthly fishing periods, monthly fishing period limits, annual landing options, and annual trip limits\\
1995 & Coastwide & Relinquishment of period limit overages\\
1996 & Coastwide & Change to species quotas,  yelloweye rockfish TAC\\
1997 & Coastwide & Initiate 5 percent quota allocation for research purposes\\
1998--1999 & Inside & 100 percent of commercial rockfish TAC allocated to the hook-and-line sector\\
1999--2000 & Coastwide & 10 percent at-sea observer coverage\\
1999--2000 & Coastwide & Selected area closures: rockfish protection areas, closed fishing areas to commercial groundfish hook-and-line gear types\\
2000--2001 & Coastwide & Allocation of rockfish species between the Pacific Halibut and hook-and-line sectors\\
2001--2002 & Inside & Limited amount of at-sea observer coverage\\
2002--2003 & Inside & 75 percent reduction of inshore rockfish TAC from 2001\\
2002--2003 & Coastwide & Expansion of catch monitoring programs\\
2002--2003 & Coastwide & Introduced 1 percent interim areas of restricted fishing, closed to all commercial groundfish fisheries\\
2004--2005 & Coastwide & RCAs expanded to 8 percent of rockfish habitats\\
2005--2006 & Inside & RCAs expanded to 28 percent of rockfish habitats\\
2005--2006 & Coastwide & Introduce groundfish license integration pilot program: 100 percent catch monitoring\\
2006--2007 & Coastwide & Introduce groundfish integrated fishery management program\\
2012 & Coastwide & Introduce trawl fishery boundaries in consultation with industry\\
2015 & Inside & Implemented Strait of Georgia and Howe Sound glass sponge reef \vphantom{1} closures\\
2015 & Inside & Implemented Strait of Georgia and Howe Sound glass sponge reef closures\\*
\end{longtable}
\clearpage
\begin{longtable}[t]{>{\raggedright\arraybackslash}p{2.5cm}>{\raggedright\arraybackslash}p{1.75cm}>{\raggedright\arraybackslash}p{7.5cm}}
\caption{\label{tab:rec-mgt-changes}Historique des changements apportés à la gestion de la pêche récréative du sébaste de 1986 à 2019.}\\
\toprule
\textbf{Année} & \textbf{Zone} & \textbf{Mesure de gestion}\\
\midrule
\endfirsthead
\caption*{}\\
\toprule
\textbf{Année} & \textbf{Zone} & \textbf{Mesure de gestion}\\
\midrule
\endhead
\
\endfoot
\bottomrule
\endlastfoot
1986 & Coastwide & 8 rockfish daily bag limit per person implemented\\
1992 & Strait of Georgia & Daily limit reduced to 5 rockfish per person in Areas 12 to 19, 28 and 29 and Subareas 20-4 and 20-7.\\
2002 & 4B & Inshore Rockfish Conservation Strategy - Daily limit reduced to 1 rockfish in Areas 12 to 19, 28 and 29 and Subareas 20-5 to 20-7.\\
2002--2007 & Coastwide & Rockfish Conservation Areas (RCAs) established - RCAs closed to fin fish harvest in recreational fishery.\\
2006 & 4B & Inshore rockfish recreational fishery closed in Areas 13 to 19, 28 and 29 from October 1.\\
2007 & 4B & Inshore rockfish recreational fishery closed October 1-May 31 in Areas 13 to 19 and Subarea 29-5. Areas 28 and 29 (except Subarea 29-5) remain closed until further notice.\\
2008--2016 & 4B & Inshore rockfish recreational fishery open May 1-September 30 in Areas 13 to 19, and Subareas 20-5 to 20-7 and 29-5. Areas 28 and 29 (except Subarea 29-5) remain closed.\\
2017 & 4B & Areas 13 to 19 and Subareas 12-1 to 12-13, 12-15 to 12-48, 20-5 to 20-7 and 29-5 open June 1 to September 30. Area 28 and 29 (except for Subarea 29-5) remain closed.\\
2019 & 4B & 1 Rockfish daily; possession limits are twice the daily limit. 0 daily + possession limit for Yelloweye Rockfish and Bocaccio. Season length May 1-October 1.\\
2019 & Coastwide & Condition of licence: "Anglers in vessels shall immediately return all rockfish that are not being retained to the water and to a similar depth from which they were caught by use of an inverted weighted barbless hook or other purpose-built descender device".\\*
\end{longtable}
\hypertarget{computational-environment}{%
\section{COMPUTATIONAL ENVIRONMENT}\label{computational-environment}}

This version of the document was generated on 2021-10-27 11:59:51 with R version 3.6.3 (2020-02-29) (R Core Team \protect\hyperlink{ref-r2019}{2019}) and R package versions:
\begin{longtable}[]{@{}lll@{}}
\toprule
Package & Version & Date\tabularnewline
\midrule
\endhead
bookdown & 0.17 & 2020-01-11\tabularnewline
cowplot & 1.0.0 & 2019-07-11\tabularnewline
csasdown & 0.0.10.9000 & 2021-05-25\tabularnewline
DLMtool & 5.4.1 & 2019-12-06\tabularnewline
dplyr & 0.8.4 & 2020-01-31\tabularnewline
gfdata & 0.0.0.9000 & 2020-03-04\tabularnewline
gfdlm & 0.0.1.9000 & 2020-03-26\tabularnewline
gfplot & 0.1.4 & 2019-12-10\tabularnewline
ggplot2 & 3.2.1 & 2019-08-10\tabularnewline
knitr & 1.28 & 2020-02-06\tabularnewline
MSEtool & 1.4.3 & 2020-01-10\tabularnewline
purrr & 0.3.3 & 2019-10-18\tabularnewline
rmarkdown & 2.1 & 2020-01-20\tabularnewline
tidyr & 1.0.2 & 2020-01-24\tabularnewline
TMB & 1.7.16 & 2020-01-15\tabularnewline
\bottomrule
\end{longtable}
The source code for this document is available at:\\
\url{https://github.com/pbs-assess/yelloweye-inside/tree/2f9a8a4}.

This document was compiled with the R package csasdown (Anderson et al. \protect\hyperlink{ref-csasdown}{2020}).

The specific versions of the primary packages used to generate this report can be viewed at:

\url{https://github.com/DLMtool/DLMtool/tree/fa971cf}\\
\url{https://github.com/tcarruth/MSEtool/tree/fa1498c}~\\
\url{https://github.com/pbs-assess/gfdata/tree/7292039}~\\
\url{https://github.com/pbs-assess/gfplot/tree/e0b36c0}~\\
\url{https://github.com/pbs-assess/gfdlm/tree/b895686}~\\
\url{https://github.com/pbs-assess/csasdown/tree/f9d5081}~\\

\vspace{4mm}

or installed via:

\texttt{\#\ install.packages(\textquotesingle{}devtools\textquotesingle{})}\\
\texttt{devtools::install\_github(\textquotesingle{}DLMtool/DLMtool\textquotesingle{},\ ref\ =\ \textquotesingle{}fa971cf\textquotesingle{})}~\\
\texttt{devtools::install\_github(\textquotesingle{}tcarruth/MSEtool\textquotesingle{},\ ref\ =\ \textquotesingle{}fa1498c\textquotesingle{})}~\\
\texttt{devtools::install\_github(\textquotesingle{}pbs-assess/gfdata\textquotesingle{},\ ref\ =\ \textquotesingle{}7292039\textquotesingle{})}~\\
\texttt{devtools::install\_github(\textquotesingle{}pbs-assess/sha\_gfplot\textquotesingle{},\ ref\ =\ \textquotesingle{}e0b36c0\textquotesingle{})}~\\
\texttt{devtools::install\_github(\textquotesingle{}pbs-assess/gfdlm\textquotesingle{},\ ref\ =\ \textquotesingle{}b895686\textquotesingle{})}~\\
\texttt{devtools::install\_github(\textquotesingle{}pbs-assess/csasdown\textquotesingle{},\ ref\ =\ \textquotesingle{}f9d5081\textquotesingle{})}~\\

\clearpage

\hypertarget{refs}{}
\leavevmode\hypertarget{ref-csasdown}{}%
Anderson, S.C., Grandin, C., Edwards, A.M., Grinnell, M.H., Ricard, D., and Haigh, R. 2020. csasdown: Reproducible CSAS reports with bookdown. R package version 0.0.8. \url{https://github.com/pbs-assess/csasdown}.

\leavevmode\hypertarget{ref-ayers2012}{}%
Ayers, C., Dearden, P., and Rollins, R. 2012. An exploration of Hul'qumi'num Coast Salish peoples' attitudes towards the establishment of no-take zones within marine protected areas in the Salish Sea, Canada. Can. Geogr. 56: 260--274.

\leavevmode\hypertarget{ref-dfo2012b}{}%
DFO. 2012. Proceedings of the Pacific region science advisory process for outside stocks of Lingcod (\emph{Ophiodon elongatus}) and the inside population of Yelloweye Rockfish (\emph{Sebastes ruberrimus}) in British Columbia, Canada, April 7-8, 2011. DFO Can. Sci. Advis. Sec. Proceed. Ser. 2011/070.

\leavevmode\hypertarget{ref-dfo2015}{}%
DFO. 2015. Evaluation of the internet recreational effort and catch (iREC) survey methods. DFO Can. Sci. Advis. Sec. Sci. Advis. Rep. 2015/059.

\leavevmode\hypertarget{ref-eckert2018}{}%
Eckert, L.E., Ban, N.C., Frid, A., and McGreer, M. 2018. Diving back in time: Extending historical baselines for yelloweye rockfish with Indigenous knowledge. Aquat. Conserv. 28(1): 158--166.

\leavevmode\hypertarget{ref-haigh2011}{}%
Haigh, R., and Yamanaka, K.L. 2011. Catch history reconstruction for rockfish (\emph{Sebastes spp.}) caught in British Columbia coastal waters. DFO Can. Tech. Rep. Fish. Aquat. Sci. 2943: viii + 124 p.

\leavevmode\hypertarget{ref-marushka2019}{}%
Marushka, L., Kenny, T.A., Batal, M., Cheung, W.W.L., Fediuk, K., Golden, C.D., Salomon, A.K., Sadik, T., Weatherdon, L.V., and Chan, H.M. 2019. Potential impacts of climate-related decline of seafood harvest on nutritional status of coastal First Nations in British Columbia, Canada. PLOS ONE 14(2): e0211473.

\leavevmode\hypertarget{ref-r2019}{}%
R Core Team. 2019. R: A language and environment for statistical computing. R Foundation for Statistical Computing, Vienna, Austria.

\leavevmode\hypertarget{ref-yamanaka2011}{}%
Yamanaka, K.L., McAllister, M.K., Olesiuk, P.F., Etienne, M.-P., Obradovich, S.G., and Haigh, R. 2011. Stock assessment for the inside population of Yelloweye Rockfish (\emph{Sebastes ruberrimus}) for British Columbia, Canada for 2010. DFO Can. Sci. Advis. Sec. Res. Doc. 2011/129. xiv + 131 p.

\leavevmode\hypertarget{ref-zetterberg2010}{}%
Zetterberg, P.R., and Carter, E.W. 2010. Strait of Georgia sport fishery creel survey statistics for salmon and groundfish 2008. DFO Can. Man. Rep. Fish. Aquat. Sci. 2929: x + 123 p.

\end{document}
